\section{Modelling Syntax}
\label{sec:notation}
\begin{center}\large\em
What do we say?
\end{center}

Like natural languages spoken by people today, programming languages have their own syntax and grammar rules.
These rules define the permissible expressions that are to be found within a language.
Fortunatly, programming languages are not as complex (i.e. are context free) as natural language and as such their syntax can be describe rather conscisly.
The most common notation used to present syntax is that of \emph{Backus-Naus Form} (BNF).
There are several varients of BNF that prove popular, for example: eBNF and aBNF.

A example of a BNF grammar to describe the syntax for a simple arithmetic language \allang{} is as follows:

\begin{bnf}
\bnfprod{AL}{%
  \bnfts{integer}
}\\
\bnfprod{AL}{%
  \bnfts{variable}
}\\
\bnfprod{AL}{%
  \bnfts{``-''}\bnfpn{AL}
}\\
\bnfprod{AL}{%
  \bnfpn{AL}\bnfts{``+''}\bnfpn{AL}
}\\
\bnfprod{AL}{%
  \bnfts{``(''}\bnfpn{AL}\bnfts{``)''}
}
\end{bnf}

\noindent
This language allows for expressions that describe integer operations addition subtraction, and modification of precedence using parentheses.
Notice how expressions are defined in terms of themselves.
Such grammars can be modelled within \idris{} as a simple data type.

\begin{code}
data Arith : Type where
  Value    : Int             -> Arith
  Var      : String -> Arith -> Arith
  Neg      : Arith           -> Arith
  Addition : Arith  -> Arith -> Arith
\end{code}

\noindent
Here we use data type constructors to model different expressions.
We also embedd simple interpretations of integers and variables.
Integers will be mapped to the \idris{} data type, and variables are represented with a name, and their value.
and symbol tables in which we can produce better representations of variables.
The syntax presented in the BNF grammar can be made more precise through use of an \emph{or} combinator.

\begin{bnf}
\bnfprod{AL}{%
  \bnfts{integer}
  \bnfor
  \bnfts{``-''}\bnfpn{AL}
  \bnfor
  \bnfpn{AL}\bnfts{``+''}\bnfpn{AL}
}
\end{bnf}

\noindent
Similar optimisations can be made to the \texttt{Arith} data type.

\begin{code}
data Arith = Value Int
           | Var String
           | Neg Arith
           | Addition Arith Arith
\end{code}

\noindent
\emph{Programming Language Theorists} (PLTs) are in the business of creating languages and BNF (and its popular variants) while descriptive and good to describe implementations can be too verbose.
PLTs use a particular variant of BNF in which the name of the language being defined (\allang{}) is replaced by a variable used to range over all the possible values of \allang{}.
The above BNF grammer would be re-expressed as follows:

\begin{bnf}
\bnfprod{$e$}{%
  \bnfts{$i$}
  \bnfor
  \bnfts{$x$}
  \bnfor
  \bnfts{$-e$}
  \bnfor
  \bnfts{$e+e$}
}
\end{bnf}

%%% Local Variables:
%%% mode: latex
%%% TeX-master: "tutorial.print"
%%% End:
