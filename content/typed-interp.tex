\section{Well-Typed $\lambda$-Calculus}
\label{sec:lambda}


This section will present a more comprehensive example that utilises the concepts seen in this tutorial.
The example will be a variant of the well-typed $\lambda$ calculus.
Specifically, this section will provide further coverage of:
\begin{compactitem}
  \item Typing Contexts; and
  \item Typing Rules.
\end{compactitem}
We will do so through consideration of nameless representations for typing environments and contexts.

\subsection{Language Syntax}
\label{sec:lambda:syntax}

We begin by introducing the syntax for our variant of the $\lambda$-calculus.
This variant of the language, which we refer to as \lamlang{}, was originally introduced in \citet{Community2014}.
We present a more specific variant in which a predefined set of binary operations are to be modelled.

\textbf{Note} the variant presented in this section does not have an \emph{as} compact representation as that presented in \citet{Community2014}.
We use a more obtuse representation on purpose.

\lamlang{} will support the following constructs:
\begin{inparaenum}
  \item Values,
  \item Variables,
  \item Binary Operators, representing arithmetic, boolean, and relational oeprations,
  \item Function application,
  \item Conditional statements; and
  \item Anonymous functions.
\end{inparaenum}
The grammar rules are:

\begin{bnf}
\bnfprod{$e$}{%
  \bnfts{$i$}
  \bnfor
  \bnfts{$x$}
  \bnfor
  \bnfts{$e\circ e$}
  \bnfor
  \bnfts{$e\;\$\;e$}
  \bnfor
  \bnfts{$e\;?\;e : e$}
  \bnfor
  \bnfts{$\lambda x:T.e$}
}\\
\bnfprod{$\circ$}{
  \bnfts{$\circ_{a}$}
  \bnfor
  \bnfts{$\circ_{r}$}
  \bnfor
  \bnfts{$\circ_{b}$}
}\\
\bnfprod{$\circ_{a}$}{
  \bnfts{$/$}
  \bnfor
  \bnfts{$*$}
  \bnfor
  \bnfts{$+$}
  \bnfor
  \bnfts{$-$}
}\\
\bnfprod{$\circ_{r}$}{
  \bnfts{$>$}
  \bnfor
  \bnfts{$<$}
  \bnfor
  \bnfts{$\equiv$}
}\\
\bnfprod{$\circ_{b}$}{
  \bnfts{$\wedge$}
  \bnfor
  \bnfts{$\vee$}
}
\end{bnf}

\noindent
In \lamlang{}:

\begin{center}
\begin{tabular}[h]{>{$}l<{$}l}
i       & are integer variables. \\
x       & represents variable names.\\
\circ   & is an infix binary operation.\\
\$      & is function application.\\
?       & is a ternary representation of a conditional statement.\\
\lambda & are anonymous functions in which the function argument $x$ has type $T$.\\
\end{tabular}
\end{center}

\noindent
The syntax for \lamlang{} can be na\"{i}vly modelled in \idris{} as follows:

\begin{code}
data Expr = Val Int
          | Var String Expr
          | Div Expr Expr | Mul Expr
          | Add Expr Expr | Sub Expr Expr
          | GT Expr Expr  | LT Expr Expr | Eqv Expr Expr
          | And Expr Expr | OR Expr Expr
          | App Expr Expr
          | If Expr Expr Expr
          | Lam Expr
\end{code}

\noindent
However, a smarter representation of \lamlang{} is to generalise binary operations according to their type such that an operation is defined as a function that is applied to two expressions.
However, a even more compact representation is to have a single generic binary expression operation.
\begin{code}
data Expr = Val Int
          | Var String Expr
          | Op (Expr -> Expr -> Expr) Expr Expr
          | App Expr Expr
          | If Expr Expr Expr
          | Lam Expr
\end{code}

\noindent
Example instances of programs written in \lamlang{} include:

\begin{align*}
  1 + 2\\
  3 > 4\\
  \backslash x.\backslash y. x + y\\
 \backslash x.(\neg x)\\
\end{align*}



\subsection{Types Revisited}
\label{sec:lambda:types}

Within \lamlang{} the types of each value will be either integers $\mathbb{Z}$, booleans $\mathbb{B}$, and functions and will be defined as transformation beteen two types.

\begin{bnf}
\bnfprod{$T$}{%
  \bnfts{$\mathbb{Z}$}
  \bnfor
  \bnfts{$\mathbb{B}$}
  \bnfor
  \bnfts{$T\rightarrow T$}
}
\end{bnf}

A function type indicates the mapping from the type of the arugment to the type of the return value.
The purpose of a type system is to allow for the type of an expression to be calculated from the expression itself.
For example, the expression $e=(1 + 2)$ will have type $\mathbb{Z}$ as the result of evaluating $e$ will be $3$ which is an integer.

These types can be modelled in \idris{} as follows:

\begin{code}
data Ty = TyValue
        | TyBool
        | TyFunc Ty Ty
\end{code}

\noindent
Again it is interesting to note how we can model in \idris{} types and grammars in similar ways.

\subsection{Types Environments Revisited}
\label{sec:lambda:typing}

When presenting the typing environment in Section~\ref{sec:typed-arith:type-env} the language did not have variables and as such a nominal ``empty'' typing environment was sufficient.
Section~\ref{sec:typed-arith-var:environments} introduced how typing environments  and contexts can be used to keep track of variables, their values, and their types.
This was illustrated using a \emph{named representation}.
In this section we will use a \emph{nameless representation} called \emph{de Bruijn indices} as an alternate representation.
\todo{Add math definition for de Bruijn incidies}.

\subsubsection{Contexts}
\label{sec:lambda:typing:contexts}

Contexts can be simply implemented as a vector of types.
As the context is to be used regularly as an implicit argument, we can define all the implementation within a \texttt{using} block.

\begin{code}
using (g:Vect n Ty)
\end{code}

\subsubsection{Membership Proof}
\label{sec:lambda:typing:membership-proof}

\emph{de Bruijn indicies} are a nameless representation in which variables are represented by a proof that states their membership of a particular context.
This proof can be represented using the data type \texttt{HasType i G T}.
Which reads a proof that variable \texttt{i} in context \texttt{G} has type \texttt{T}.
We can define it as follows:

\begin{code}
data HasType : (i : Fin n) -> Vect n Ty -> Ty -> Type where
    Recent : HasType FZ (t :: g) t
    Next   : HasType k g t -> HasType (FS k) (u :: g) t
\end{code}

\noindent
\texttt{HasType} is an inductive style proof.
The constructor \texttt{Recent} can be seen as the base case which states that the most recently defined variable is well typed.
The constructor \texttt{Next n} is the inductive case that states if the $n^{th}$ most recently defined variable is well-typed, then so is the $n+1^{th}$ variable.
With this \emph{proof}, \texttt{Recent} represents the most recently defined variables, and \texttt{Next Recent} to refer to the next variable and so on.

\subsubsection{Environments}
\label{sec:lambda:typying:envs}

To keep track of the types in scope, we need an environment.
Our definition of environment changes little from Section~\ref{sec:typed-arith-var:environments}.
We drop the variable name value pairings and just store the values.

\begin{code}
data Env : Vect n Ty -> Type where
    Nil  : Env Nil
    (::) : interpTy a -> Env g -> Env (a :: g)
\end{code}

\noindent
As with our definition of the context, we store the most recently defined variable at the head of the list.
We also abuse the list construct so that the list syntax from \idris{} can be used.

We can also define a function \texttt{lookup} that when given a proof that a variable is defined within the context, a value can be extracted from the environment.

\begin{code}

lookup : HasType i g t -> Env g -> interpTy t
lookup Recent   (x :: xs) = x
lookup (Next k) (x :: xs) = lookup k xs
\end{code}

\subsection{Typing Rules}
\label{sec:lambda:rules}

Here we define the rules that allow us to construct \emph{well-typed} expressions.

\subsubsection{Values}
\label{sec:type:rules:values}
Here we define integer and boolean values and assign them a type.
\begin{prooftree}
\AxiomC{}
\LeftLabel{Integers}
\UnaryInfC{$\Gamma\vdash i : \mathbb{Z} $}
\end{prooftree}

\begin{prooftree}
\AxiomC{}
\LeftLabel{Booleans}
\UnaryInfC{$\Gamma\vdash b : \mathbb{B} $}
\end{prooftree}

\subsubsection{Arithmetic Operations}
\label{sec:type:rules:arith}

Arithmetic operations have the following types.

\begin{prooftree}
\AxiomC{$\Gamma\vdash e_1 : \mathbb{Z}$}
\AxiomC{$\Gamma\vdash e_2 : \mathbb{Z}$}
\LeftLabel{Division}
\BinaryInfC{$\Gamma\vdash e_1/e_2 : \mathbb{Z}$}
\end{prooftree}

\begin{prooftree}
\AxiomC{$\Gamma\vdash e_1 : \mathbb{Z}$}
\AxiomC{$\Gamma\vdash e_2 : \mathbb{Z}$}
\LeftLabel{Multiplication}
\BinaryInfC{$\Gamma\vdash e_1*e_2 : \mathbb{Z}$}
\end{prooftree}

\begin{prooftree}
\AxiomC{$\Gamma\vdash e_1 : \mathbb{Z}$}
\AxiomC{$\Gamma\vdash e_2 : \mathbb{Z}$}
\LeftLabel{Addition}
\BinaryInfC{$\Gamma\vdash e_1+e_2 : \mathbb{Z}$}
\end{prooftree}

\begin{prooftree}
\AxiomC{$\Gamma\vdash e_1 : \mathbb{Z}$}
\AxiomC{$\Gamma\vdash e_2 : \mathbb{Z}$}
\LeftLabel{Subtraction}
\BinaryInfC{$\Gamma\vdash e_1-e_2 : \mathbb{Z}$}
\end{prooftree}

\subsubsection{Boolean Operations}
\label{sec:types:rules:boolean}

Boolean operations can only work on bools.

\begin{prooftree}
\AxiomC{$\Gamma\vdash e_1 : \mathbb{B}$}
\AxiomC{$\Gamma\vdash e_2 : \mathbb{B}$}
\LeftLabel{And}
\BinaryInfC{$\Gamma\vdash e_1\wedge e_2 : \mathbb{B}$}
\end{prooftree}

\begin{prooftree}
\AxiomC{$\Gamma\vdash e_1 : \mathbb{B}$}
\AxiomC{$\Gamma\vdash e_2 : \mathbb{B}$}
\LeftLabel{Or}
\BinaryInfC{$\Gamma\vdash e_1\vee e_2 : \mathbb{B}$}
\end{prooftree}

\begin{prooftree}
\AxiomC{$\Gamma\vdash e : \mathbb{B}$}
\LeftLabel{Negation}
\UnaryInfC{$\Gamma\vdash \neg e : \mathbb{B}$}
\end{prooftree}

\subsubsection{Relational Operations}
\label{sec:types:rules:relation}

Numerical comparison operations result in boolean results.

\begin{prooftree}
\AxiomC{$\Gamma\vdash e_1 : \mathbb{Z}$}
\AxiomC{$\Gamma\vdash e_2 : \mathbb{Z}$}
\LeftLabel{Greater Than}
\BinaryInfC{$\Gamma\vdash e_1 > e_2 : \mathbb{B}$}
\end{prooftree}

\begin{prooftree}
\AxiomC{$\Gamma\vdash e_1 : \mathbb{Z}$}
\AxiomC{$\Gamma\vdash e_2 : \mathbb{Z}$}
\LeftLabel{Less than}
\BinaryInfC{$\Gamma\vdash e_1< e_2 : \mathbb{B}$}
\end{prooftree}

\begin{prooftree}
\AxiomC{$\Gamma\vdash e_1 : \mathbb{Z}$}
\AxiomC{$\Gamma\vdash e_2 : \mathbb{Z}$}
\LeftLabel{Equality}
\BinaryInfC{$\Gamma\vdash e_1\equiv e_2 : \mathbb{B}$}
\end{prooftree}


\subsubsection{Variables}
\label{sec:types:rules:variables}
Variables can either be booleans or integers.
\begin{prooftree}
\AxiomC{$(x,T)\in\Gamma$}
\AxiomC{$T:\mathbb{Z}$}
\LeftLabel{Integer Variables}
\BinaryInfC{$\Gamma\vdash i : \mathbb{Z} $}
\end{prooftree}

\begin{prooftree}
\AxiomC{$(x,T)\in\Gamma$}
\AxiomC{$T:\mathbb{B}$}
\LeftLabel{Boolean Variables}
\BinaryInfC{$\Gamma\vdash b : \mathbb{B} $}
\end{prooftree}

\subsubsection{Functions}
\label{sec:types:rules:functions}

\begin{prooftree}
\AxiomC{$\Gamma\vdash e_1 : \mathbb{B}$}
\AxiomC{$\Gamma\vdash e_2 : T$}
\AxiomC{$\Gamma\vdash e_3 : T$}
\LeftLabel{Conditionals}
\TrinaryInfC{$\Gamma\vdash (e_1?e_2:e_3): T$}
\end{prooftree}

\begin{prooftree}
\AxiomC{$\Gamma,x:T_1\vdash e : T_2$}
\LeftLabel{Anonymous Functions}
\UnaryInfC{$\Gamma\vdash (\lambda x:T_1.e) : (T_1\rightarrow T_2)$}
\end{prooftree}

\begin{prooftree}
\AxiomC{$\Gamma\vdash e_1 : T\rightarrow\alt{T}$}
\AxiomC{$\Gamma\vdash e_2 : T$}
\LeftLabel{Application}
\BinaryInfC{$\Gamma\vdash e_1 \$ e_2 : \alt{T}$}
\end{prooftree}

\subsection{Model}
\label{sec:lambda:model}
\begin{code}
data Expr : Vect n Ty -> Ty -> Type where
    Var : HasType i g t   -> Expr g t
    Val : (x : Int)       -> Expr g TyInt
    Lam : Expr (a :: g) t -> Expr g (TyFun a t)
    App : Expr g (TyFun a t) -> Expr g a -> Expr g t
    Op  : (interpTy a -> interpTy b -> interpTy c)
            -> Expr g a
            -> Expr g b
            -> Expr g c
    If  : Expr g TyBool
          -> Lazy (Expr g a)
          -> Lazy (Expr g a)
          -> Expr g a
\end{code}

\subsection{Denotational Semantics}
\label{sec:lambda:semantics}

The purpose of this section is to provide an \emph{interpretation} of \lamlang{} to \idris{}.
The next section will illustrate how an interpreter can be constructed in \idris{} to run our \lamlang{} programs.

\subsubsection{Values \& Variables}
\label{sec:semantics:variables}
Values are translated directly into their \idris{} representation.
Variables are special\ldots

\begin{center}
\begin{tabularx}{0.8\textwidth}{>{$}r<{$}>{\ttfamily}X}
\interpB{i}     =& \\
\interpB{x}     =& \\
\end{tabularx}
\end{center}

\subsubsection{Types}
\label{sec:semantics:types}

Types are mapped to their \idris{} equivalents.

\begin{center}
\begin{tabularx}{0.8\textwidth}{>{$}r<{$}>{\ttfamily}X}
\interpB{}     =& \\
\end{tabularx}
\end{center}

\subsection{Binary Operators}
\label{sec:semantics:binary-operations}

All binary operations are mapped directly to their \idris{} versions.

\begin{center}
\begin{tabularx}{0.8\textwidth}{>{$}r<{$}>{\ttfamily}X}
\interpB{}     =& \\
\end{tabularx}
\end{center}

\subsection{Functions \& Conditionals}
\label{sec:semantics:functions}

Conditionals are mapped to the \idris{} ifthenelse construct, and anonymous functions to \idris{}'s own anonymous functions.

\begin{center}
\begin{tabularx}{0.8\textwidth}{>{$}r<{$}>{\ttfamily}X}
\interpB{}     =& \\
\end{tabularx}
\end{center}

%%% Local Variables:
%%% mode: latex
%%% TeX-master: "../tutorial.screen"
%%% End:
