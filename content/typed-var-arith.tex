\section{Well-Typed Arithmetic with Variables}
\label{sec:typed-arith-var}

In this section we will extend our well-typed arithmetic language to include the use of free variables.
Specifically, this section will provide further coverage on:
\begin{compactitem}
  \item Typing Contexts; and
  \item Typing Rules.
\end{compactitem}

\subsection{Language Syntax}
\label{sec:typed-arith-var:syntax}

We extend the syntax of our language to include variables:

\begin{bnf}
\bnfprod{$e$}{%
  \bnfts{$i$}
  \bnfor
  \bnfts{$x$}
  \bnfor
  \bnfts{$-e$}
  \bnfor
  \bnfts{$e+e$}
  \bnfor
  \bnfts{$e-e$}
  \bnfor
  \bnfts{$e/e$}
  \bnfor
  \bnfts{$e*e$}
}
\end{bnf}

\noindent
Variables will be represented as $x$.
Before the resulting \idris{} representation can be given we must first re-examine our notion of typing environments and introduce typing contexts.

\subsection{Types}
\label{sec:typed-arith-var:types}

When presenting the typing environment in Section~\ref{sec:typed-arith:type-env} the language did not have variables and as such a nominal ``empty'' typing environment was sufficient.
However, we have now introduced variables.
Working with variables is problematic in that variables are  mutable and the type of a variable will not be known \emph{a priori}, and may change during program execution\footnote{Well with this particular language, variables will always have the same type.}.
To address this issue we partially introduced the concept of a \emph{typing environment} in Section~\ref{sec:typed-arith:type-env}.

Traditionally, typing environments are denoted by the greek letter $\Gamma$.
A typing environment is defined as a set that contains tuples of variable values and their types: $(x,T)$.
Type environments can be extended, and the addition of new variables to the environment may result in old definitions being removed.

When working with typing environments it is important to consider the list of available types separately from their values.
This separation allows us to consider the types for particular contexts outside what their values actually are.

\subsubsection{Typing Contexts}
\label{sec:typed-arith-var:contexts}

We define a \emph{typing context} as the pairing between a representation of a variable and its type.
We first construct a generic type alias \texttt{Context ty}, that represents a typing context with type \texttt{ty}.

\begin{code}
Context : Type -> Type
Context ty = List (String, ty)
\end{code}

\noindent
Using this alias we can then create typing environments of type \texttt{Context ty}.
For example type environments using our language can be modelled as

\begin{code}
example_context : Context ArithTy
example_context = [("foo", TyValue), ("bar", TyValue)]
\end{code}

\noindent
\textbf{Note} Here we are purposefully using a \emph{named representation} and keeping track of variables using their actual name.
This makes it easier to understand what is going on within contexts.
An alternate would be to use a \emph{nameless representation}, in our larger case study (Section~\ref{}) we will show how a nameless representation using \emph{de Bruijn} indices works.

Before type environments can be used in anger we define several operations that operate on $\Gamma$.

\paragraph{Removing}
\label{sec:typed-arith-var:types:remove}

The first operation will update $\Gamma$ by removing all references to $x$.
\[
\Gamma\backslash x
\]
\noindent
This operation can be implemented as follows:
\begin{code}
remove : (String, ty) -> Context ty -> Context ty
remove e es = deleteBy (\(x,y),(c,d) => x==c) e es
\end{code}

\paragraph{Extending}
\label{sec:typed-arith-var:types:extend}

The second operations is used to extend $\Gamma$ with a variable $x$ of type $T$.
This operation will overide previous $x$.
\[
\Gamma,x:T = (\Gamma\backslash x)\cup\{x,T\}
\]
\noindent
Which can be representened as:
\begin{code}
extend : (String, ty) -> Context ty -> Context ty
extend e es = e :: (remove e es)
\end{code}

\paragraph{Searching}
\label{sec:typed-arith-var:types:lookup}

For completness, the final operation defined is used to search the typying environment to extract the type for a specified variable.
\[
\mathsf{lookup}(\Gamma,x:T) = \text{??}
\]
\noindent
This can be implemented as:
\begin{code}
lookup : String -> Context ty -> Maybe ty
lookup = List.lookup
\end{code}

\paragraph{Improving \idris{} Implementation}
\label{sec:typed-arith-var:types:classes}

We can enforce our formal models of typing contexts in \idris{} better by creating a type class that groups our default implementations together as follows:

\begin{code}
class (Eq ty) => TypingContext ty where
  remove : (String, ty) -> Context ty -> Context ty
  extend : (String, ty) -> Context ty -> Context ty
  lookup : String       -> Context ty -> Maybe ty

  remove e es = deleteBy (\(x,y),(c,d) => x==c) e es
  extend e es = e :: (remove e es)
  lookup = List.lookup
\end{code}

\paragraph{Examples}
\label{sec:typed-arith-var:types:examples}

Explaining changes to a typing context is difficult to observe with type systems that only have a single type.
We further motivate the need for typing contexts by using an alternate type system in which we introduce an additional type: $\mathbb{B}$ for expressions of type boolean.
Such boolean types will have a type constructor $TyBool$.
Examples of typing contexts and their operations are as follows:

\[
\Gamma_{1}=\{(foo,\mathbb{Z}),(g,\mathbb{B}\}
\]

\begin{code}
ctxt : Context ArithTy
ctxt = [("foo", TyValue), ("g", TyBool)]
\end{code}

\noindent
$\Gamma_{1}$ can be updated as follows:
\[
\Gamma_{1},g:\mathbb{Z} =\{(foo,\mathbb{Z}),(g,\mathbb{Z})\}
\]
\begin{code}
env' : TypingContext ArithTy => Context ArithTy
env' = extend ("g", TyValue) ctxt
\end{code}
\noindent
and:
\[
\Gamma_{1},f:\mathbb{B}=\{(foo,\mathbb{Z}),(g,\mathbb{Z}),(f,\mathbb{B})\}
\]
\begin{code}
env'' : TypingContext ArithTy => Context ArithTy
env'' = extend ("f", TyBool) env'
\end{code}

\subsubsection{Typing Environments }
\label{sec:typed-arith-var:environments}

Now that we have defined a typing context we look to define a typing environment.
Recall that our typing environment is defined a set of tuples containing variable values and their types: $(x,T)$.
This can be represented in \idris{} as a variant of a list in which the data type contains the typing context:

\begin{code}
data Env : Context ArithTy -> Type where
  Nil  : Env Nil
  (::) : (e : (String, Int)) -> Env g
       -> Env (extend (fst e, TyValue) g)
\end{code}

\noindent
We also construct a complementary function that allows us to extract the value of a variable from the environment.

\begin{code}
getValue : String -> Env g -> Int
getValue n Nil           = 0
getValue n ((a,v) :: xs) = case n == a of
  True => v
  False => fromEnv n xs
\end{code}

\subsection{Additional Typing Rules}
\label{sec:typed-arith-var:typing-rules}

With our new knowledge of typing environments we can now update our list of typing rules to include variables:

\begin{prooftree}
\AxiomC{$(x,T)\in\Gamma$}
\AxiomC{$T:\mathbb{Z}$}
\LeftLabel{Variables}
\BinaryInfC{$\Gamma\vdash i : \mathbb{Z}$}
\end{prooftree}

\noindent
In our language, a variable will be associated with a type $T$ if that variable exists within the typing environment.
That is if the result of a look-up in the typing context returns the type and not \texttt{Nothing}.
For example:

\begin{code}
ctxt : Context ArithTy
ctxt = [("foo", TyValue), ("g", TyBool)]

test : Bool
test = (Just TyValue) == lookup "foo" ctxt
\end{code}

\noindent
However, as our language only has a single type we can skip this step\footnote{Or maybe not\ldots}.

\subsection{New Model}
\label{sec:typed-arith-var:model}

Armed with our knowledge of typing contexts and extended set of typing rules, we can now proceed to define new version of the \texttt{Arith} data type in which we embed the typing context $\Gamma$ within each expression.

\begin{code}
data Arith : Context ArithTy -> ArithTy -> Type where
  Val : Int                                -> Arith g TyValue
  Var : (n : String)                       -> Arith g TyValue
  Neg : Arith g TyValue                    -> Arith g TyValue
  Add : Arith g TyValue -> Arith g TyValue -> Arith g TyValue
  Sub : Arith g TyValue -> Arith g TyValue -> Arith g TyValue
  Div : Arith g TyValue -> Arith g TyValue -> Arith g TyValue
  Mul : Arith g TyValue -> Arith g TyValue -> Arith g TyValue
\end{code}

\subsection{Updated Semantics}
\label{sec:typed-arith-var:semantics}

And redefine our interpreter to work with new definition of \texttt{Arith}.

\begin{code}
interp : Env g -> Arith g TyValue -> Int
interp env (Val x)    = x
interp env (Var n)    = getValue n env
interp env (Neg x)    = (-1) * (interp env x)
interp env (Add x y)  = (interp env x) + (interp env y)
interp env (Sub x y)  = (interp env x) - (interp env y)
interp env (Div x y)  = (interp env x) `div` (interp env y)
interp env (Mul x y)  = (interp env x) * (interp env y)
\end{code}

\subsection{Executing Programs}
\label{sec:typed-arith-var:running}

We can now proceed to execute programs.
As our programs use free variables, variables and their values must be defined within the environment.
For example we define the program: $1 + x$, and then execute the program with $x=3$.

\begin{code}
main : IO ()
main = do
  print $ interp [("x",3)] (Add (Val 1) (Var "x"))
\end{code}

%%% Local Variables:
%%% mode: latex
%%% TeX-master: "../tutorial.print"
%%% End:
