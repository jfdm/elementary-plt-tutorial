\section{Introduction}
\label{sec:intro}

At the \emph{original} time of writing this tutorial I was relearning/learning elementary concepts in \emph{Programming Language Theory} that went well-beyond grammars.
Notably: I was looking into topics such as type systems, denotational semantics, and operational semantics.
Given my interest in practical uses of dependent types, in which I use the \idris{} programming language for my research, I decided to code-up these theoretical knowns as \idris{} code.
Moreover, when attemping to understand the case study for the \emph{Well-Typed} interpreter from the \idris{} tutorial \cite[Section~6]{Community2014} some of the concepts presented and encoded require some thought for those not versed in programming language theory.
Even the elementary topics.
The \emph{Well-Typed Interpreter} presents an interpreter for the Well-Typed $\lambda$ calculus.

Fortunatly, I was directed to the excellent blog post \citetitle{Siek2012ccp} by \citeauthor{Siek2012ccp} \cite{Siek2012ccp} by a colleague.
In this blog, \citeauthor{Siek2012ccp} provides the `crash course' through introduction of the \allang{}.
However, these concepts were not necessarily enforced through implementations.
You might as well put theory into practise, no?

\subsection{Contribution}
\label{sec:intro:contribution}

This tutorialss' contributions are summarised as follows:

\paragraph{Contribution 1}
I present my notes on marrying the theory and practise for the \allang{}, illustrating how the theoretical concepts presented in \citet{Siek2012ccp} can be mapped to an interpreter written in \idris{}.

\paragraph{Contribution 2}
I present the \emph{Well-Typed} interpreter from the \idris{} tutorial, together with additional material explaining the underlying formal methods used.

\subsection{Intended Audience}
\label{sec:intro:audience}

This tutorial is presented as my notes\footnote{There will be mistakes, missing information, typos, and \emph{Citations Required}.}.
The intended audience are those who are not familiar with elementary concepts in programming language theory, and have \emph{some} familiarity with \idris{}.
The material is presented in a possible more verbose fashion than required.
This is to be explicit about what is going on.
If you are not familiar with \idris{}, I recommend the \citetitle{Community2014}\cite{Community2014}.
