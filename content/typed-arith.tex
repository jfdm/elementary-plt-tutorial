\section{Well-Typed Arithmetic}
\label{sec:typed-arith}

\noindent
This section will introduce some of the basic concepts in programming language theory.
We will consider the construction of a simple language that performs integer arithmetic.
This section will cover:

\begin{compactitem}
\item Grammars
\item Basic Type Theory
  \begin{compactitem}
  \item Types
  \item Simple Typing Contexts
  \item Simple Typing Rules
  \end{compactitem}
\item Denotational Semantics.
\end{compactitem}

\subsection{Modelling Language Syntax}
\label{sec:typed-arith:syntax}

Like natural languages spoken by people today, programming languages have their own syntax and grammar rules.
These rules define the permissible expressions that are to be found within a language.
Fortunatly, programming languages are not as complex (i.e. are context free) as natural language and as such their syntax can be describe rather conscisly.
The most common notation used to present syntax is that of \ac{bnf}.
There are several variants of \ac{bnf} that prove popular, for example: \ac{ebnf} \cite{iso14977} and \ac{abnf} \cite{rfc5234}.

A example of a \ac{bnf} grammar to describe the syntax for a simple arithmetic language is as follows:

\begin{bnf}
\bnfprod{AL}{%
  \bnfts{integer}
}\\
\bnfprod{AL}{%
  \bnfts{``-''}\bnfpn{AL}
}\\
\bnfprod{AL}{%
  \bnfpn{AL}\bnfts{``+''}\bnfpn{AL}
}\\
\bnfprod{AL}{%
  \bnfpn{AL}\bnfts{``-''}\bnfpn{AL}
}\\
\bnfprod{AL}{%
  \bnfpn{AL}\bnfts{``*''}\bnfpn{AL}
}\\
\bnfprod{AL}{%
  \bnfpn{AL}\bnfts{``/''}\bnfpn{AL}
}\\
\end{bnf}

\noindent
This language allows for expressions that describe several binary operations on integers, and a single unary operation.
Notice how expressions are defined in terms of themselves.
Such grammars can be modelled within \idris{} as a simple data type.

\begin{code}
data Arith : -> Type where
  Val : Int            -> Arith
  Neg : Arith          -> Arith
  Add : Arith -> Arith -> Arith
  Sub : Arith -> Arith -> Arith
  Div : Arith -> Arith -> Arith
  Mul : Arith -> Arith -> Arith
\end{code}

\noindent
Here we use data type constructors to model different expressions.
We also embedd an interpretations of integers.
Integers will be mapped to the \idris{} data types.
The syntax presented in the \ac{bnf} grammar can be made more precise through use of an \emph{or} combinator.

\begin{bnf}
\bnfprod{AL}{%
  \bnfts{integer}
  \bnfor
  \bnfts{``-''}\bnfpn{AL}
  \bnfor
  \bnfpn{}ops
}\\
\bnfprod{ops}{%
  \bnfpn{AL}\bnfts{``+''}\bnfpn{AL}
  \bnfor
  \bnfpn{AL}\bnfts{``-''}\bnfpn{AL}
  \bnfor
  \bnfpn{AL}\bnfts{``/''}\bnfpn{AL}
  \bnfor
  \bnfpn{AL}\bnfts{``*''}\bnfpn{AL}
}
\end{bnf}

\noindent
Similar optimisations can be made to the \texttt{Arith} data type.

\begin{code}
data Arith = Val Int
           | Neg Arith
           | Add Arith Arith
           | Sub Arith Arith
           | Div Arith Arith
           | Mul Arith Arith
\end{code}

\noindent
\emph{Programming Language Theorists} (PLTs) are in the business of creating languages and \ac{bnf} (and its popular variants) while descriptive and good to describe implementations can be too verbose.
PLTs use a particular variant of \ac{bnf} in which the name of the language being defined (\allang{}) is replaced by a variable used to range over all the possible values of \allang{}.
The above \ac{bnf} grammer would be re-expressed as follows:

\begin{bnf}
\bnfprod{$e$}{%
  \bnfts{$i$}
  \bnfor
  \bnfts{$-e$}
  \bnfor
  \bnfts{$e+e$}
  \bnfor
  \bnfts{$e-e$}
  \bnfor
  \bnfts{$e/e$}
  \bnfor
  \bnfts{$e*e$}
}
\end{bnf}

\subsection{Basic Types}
\label{sec:typed-arith:types}

\glsplural{grammar} helps us to define the syntax of our language, essentially \emph{what we say}.
\gls{tysys} helps us define \emph{a means to know what we say is correct}.
Types are used to helps us take note of the \emph{kind} of objects that will exist when the program is executed.
These objects will represent the types in our type system.
The set of types in a type system is often represented as $T$.
For the language presented earlier there is only one type of object in play integers $\mathbb{Z}$.

\begin{bnf}
\bnfprod{$T_{Arith}$}{%
  \bnfts{$\mathbb{Z}$}
}
\end{bnf}

\noindent
When modelling types in \idris{} a data \texttt{Ty} type can be constructed.
The type object for the \allang{} is simply:

\begin{code}
data ArithTy = TyValue
\end{code}

\noindent
The purpose of a type system is to allow for the type of an expression to be calculated from the expression itself.
Continuting with the \allang{}, for example, the expression $e=(1 + 2)$ will have type $\mathbb{Z}$ as the result of evaluating $e$ will be $3$ which is an integer.

Type systems are defined using relations that will allow for the pairing of expressions to types.
We call this relation \textsf{WellTyped}, and will contain only: \emph{correctly typed expresions paired with their type}.
Thus:
\[
((1+2), Int)\in\mathsf{WellTyped}\\
(\text{``Bob''},Int)\notin\mathsf{WellTyped}
\]

\noindent
Taking our \idris{} translations \textsf{WellTyped} can be represented as a list of expression type pairs.

\begin{code}
welltyped : List (Arith, ArithTy)
welltyped = [(Add (Val 1) (Val 2), TyValue)]
\end{code}

\noindent
Note we do not have a means (yet) to ensure that only well-typed expressions are constructed.
These are typing rules and are introduced in Section~\ref{sec:typed-arith:rules}.
Before we can specify typing rules, we need to first introduce the notion of \emph{Typing Environments}.

\subsection{Typing Environments}
\label{sec:typed-arith:type-env}

When working with languages keeping track of what elements in the language have what types is important.
For simple languages, such as the one introduced in this section, there is no need: All expressions have the same type.
However, in languages with variables, simple relations are not sufficient.
\emph{Typing Environments} are used to keep track of local variables and their types, and are explained in more depth in the next section.

Traditionally, typing environments are denoted by the greek letter $\Gamma$.
For the purposes of this section, modelling complete typing environments is not required and our typing environment $\Gamma$ just needs to nominally exist.

Using this notion of a type environment we can improve our definition of $\mathsf{WellTyped}$ to include triples in the form of $(\Gamma,e,T)$.
The $\mathsf{WellTyped}$ set will contain expressions that have a type $T$ derived from a local context $\Gamma$.
\[
(\Gamma,e,T)\in\mathsf{WellTyped}
\]
\noindent
To save on typing, the short hand $\Gamma\vdash e:T$ is used.
As our typing environment is empty the empty set symbol is used instead of $\Gamma$.

\subsection{Typing Rules}
\label{sec:typed-arith:rules}

Types and typing environments act as building blocks to help us construct well-typed programs.
To construct the set of relations for \textsf{WellTyped}, we need to define \emph{Typing Rules} that specify how expressions are typed and how types interact when expressions are combined.

\subsubsection{What are Typing Rules?}
\label{sec:typed-arith:rules:what}

Typing rules are a series of judgments that work in a particular context, with the top line defining the inputs and the bottom line the result.
The language in consideration in this section will have the following small set of typing rules.
Integers are given the type $\mathbb{Z}$.

\begin{prooftree}
\AxiomC{}
\LeftLabel{Integers}
\UnaryInfC{$\emptyset\vdash i : \mathbb{Z} $}
\end{prooftree}

\noindent
Negation is only applied to integers.
\begin{prooftree}
\AxiomC{$\emptyset\vdash e : \mathbb{Z}$}
\LeftLabel{Addition}
\UnaryInfC{$\emptyset\vdash -e : \mathbb{Z}$}
\end{prooftree}

\noindent
Operations only work with integers:

\begin{prooftree}
\AxiomC{$\emptyset\vdash e_1 : \mathbb{Z}$}
\AxiomC{$\emptyset\vdash e_2 : \mathbb{Z}$}
\LeftLabel{Addition}
\BinaryInfC{$\emptyset\vdash e_1+e_2 : \mathbb{Z}$}
\end{prooftree}

\begin{prooftree}
\AxiomC{$\emptyset\vdash e_1 : \mathbb{Z}$}
\AxiomC{$\emptyset\vdash e_2 : \mathbb{Z}$}
\LeftLabel{Subtraction}
\BinaryInfC{$\emptyset\vdash e_1-e_2 : \mathbb{Z}$}
\end{prooftree}

\begin{prooftree}
\AxiomC{$\emptyset\vdash e_1 : \mathbb{Z}$}
\AxiomC{$\emptyset\vdash e_2 : \mathbb{Z}$}
\LeftLabel{Multiplication}
\BinaryInfC{$\emptyset\vdash e_1*e_2 : \mathbb{Z}$}
\end{prooftree}

\begin{prooftree}
\AxiomC{$\emptyset\vdash e_1 : \mathbb{Z}$}
\AxiomC{$\emptyset\vdash e_2 : \mathbb{Z}$}
\LeftLabel{Division}
\BinaryInfC{$\emptyset\vdash e_1/e_2 : \mathbb{Z}$}
\end{prooftree}

\subsubsection{Modelling Typing Rules}
\label{sec:typed-arith:rules:modelling}

In non dependently typed languages we can \emph{model} typing rules through pattern matching.
For example here is na\"{i}ve implementation of \texttt{Addition}:
\begin{code}
addition : Arith -> Arith -> Maybe Arith
addition (Value a) (Value b) = Just (Value (a + b))
addition _         _         = Nothing
\end{code}

\noindent
However, there are two problems with this implementation.
First, this might be a bad implementation through the use of \texttt{Maybe}\footnote{
My programming foo is not strong with this.}
Secondly, it is rather verbose, and requires the creation of special functions to construct expressions; we have data constructors for that.

Here is a somewhat better implementation of the typing rules.
We leverage our ability to use dependent types, and embedd the typing rules directly within the constructors of the data type \texttt{Arith}

\begin{code}
data Arith : ArithTy -> Type where
  Value    : Int                            -> Arith TyValue
  Var      : String        -> Arith TyValue -> Arith TyValue
  Neg      : Arith TyValue                  -> Arith TyValue
  Addition : Arith TyValue -> Arith TyValue -> Arith TyValue
\end{code}

\noindent
Just look at that compact representation and emebedding of the typing rules.
Dependent types cool!


\paragraph{Note}
An alternate means to model this simple typed Arithmetic language is to introduce types for the operations.
This will allow for a more compact and stronger definition.
We leave this as an exercise for the reader.

\subsection{Denotational Semantics}
\label{sec:typed-arith:semantics}

So far we have introduced: how to model syntax, represent types, and enforce typing rules.
In this section we introduce the notion of \emph{Denotational Semantics} to provide an \emph{interpretation} of the language into \idris{}.
Denotational semantics is a technique that allows us to define the semantics of a language's expressions using another base notation, usually set-theory.
However, \idris{} is essentially applied maths\ldots and I am too lazy to provide a full denotational semantics for the language\footnote{This is left as an exercise for the author\ldots}.
Instead  we will provide an interpretation of in \idris{} such that we can build an interpreter that will allow us to interpret and execute programs.
Here the notation $\interpB{e}$ will be used to denote the interpretation of an element.

Alternatively, the definition of a language's semantics can be defined computationally using operational semantics\footnote{\textsc{Ibid}}.

\subsubsection{Interpreting Types}
\label{sec:typed-arith:semantics:types}

We begin by providing an interpretation of the types $T_{Arith}$:

\begin{center}
\begin{tabularx}{0.8\textwidth}{>{$}r<{$}>{\ttfamily}X}
\interpB{\mathbb{Z}}=& Int \\
\end{tabularx}
\end{center}

\noindent
A type interpreter can be written as follows:

\begin{code}
interpTy : ArithTy -> Type
interpTy TyValue = Int
\end{code}

\subsubsection{Expressions}
\label{sec:typed-arith:semantics:exrs}

Now we look to interpret expressions.

\begin{center}
\begin{tabularx}{0.8\textwidth}{>{$}r<{$}>{\ttfamily}X}
\interpB{i}     =& i\\
\interpB{-e}    =& (-1) * $\interpB{e}$\\
\interpB{x + y} =& $\interpB{x}$ + $\interpB{x}$ \\
\interpB{x - y} =& $\interpB{x}$ - $\interpB{x}$ \\
\interpB{x / y} =& $\interpB{x}$ `div` $\interpB{x}$ \\
\interpB{x * y} =& $\interpB{x}$ * $\interpB{x}$ \\
\end{tabularx}
\end{center}

\noindent
Raw values are directly translated into \idris{} values with type \texttt{Int}.
Negative numbers are interpreted expressions multiplied by $-1$.
Finally, the binary operations are mapped directly to their \idris{} equivalents.
Our interpreter for the language is thus constructed as follows:

\begin{code}
interp : Arith t -> interpTy t
interp (Val x)   = x
interp (Neg x)   = (-1) * (interp x)
interp (Add x y) = (interp x) + (interp y)
interp (Sub x y) = (interp x) - (interp y)
interp (Div x y) = (interp x) `div` (interp y)
interp (Mul x y) = (interp x) * (interp y)
\end{code}

\noindent
Not the similarities between the formalised semantics and representation in \idris{}.

\subsection{Running the Interpreter.}
\label{sec:running-interpreter}

Now that we have constructed the interpreter for the language we now can use the \texttt{interp} function to execute expressions.

\begin{code}
main : IO ()
main = do
  let expr = (Add (Val 1) (Val 2))
  print $ interp expr
\end{code}

%%% Local Variables:
%%% mode: latex
%%% TeX-master: "../tutorial.print"
%%% End:
