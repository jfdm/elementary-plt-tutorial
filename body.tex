
\input{conf.ltx}

\newcommand{\version}{\gitVtagn}

\title{Introduction to some \emph{Formal Methods} using \idris{}}
\author{Jan de Muijnck-Hughes}
\date{\origdate\today}
\addbibresource{biblio.bib}
\loadglsentries[type=\acronymtype]{./acronyms.gloss}
\loadglsentries{./glossary.gloss}
\makeglossaries

\begin{document}

\maketitle%
\begin{abstract}
This tutorial will provide an introduction to some elementary topics in programming language theory.

\begin{itemize}
\item Namely: grammars, type systems, and denotational semantics.
\item We will do this through construction of interpreters for two simple programming languages: \allang{}, and the well-typed lambda calculus---\lamlang{}.
The dependently typed programming language \idris{} will be used to introduce these concepts \cite{}.
\begin{itemize}
\item \allang{} was orginally described formally in a blog post entitled \citetitle{Siek2012ccp} \cite{Siek2012ccp}.
Here we present the formalisations together with an interpreter written in \idris{}.

\item The interpreter for the well-typed $\lambda$ calculus was originally presented in the offical tutorial for \idris{} \cite{Community2014}.
Here we present the interpreter with some additional material explaining the underlying formal methods.
\end{itemize}
\item More information over Idris can be found online at: \url{http://www.idris-lang.org}
\end{itemize}
\end{abstract}

\newpage
\tableofcontents
\newpage

\section{Introduction}
\label{sec:intro}

At the \emph{original} time of writing this tutorial I was relearning/learning elementary concepts in \emph{Programming Language Theory} that went well-beyond grammars.
Notably: I was looking into topics such as type systems, denotational semantics, and operational semantics.
Given my interest in practical uses of dependent types, in which I use the \idris{} programming language for my research, I decided to code-up these theoretical knowns as \idris{} code.
Moreover, when attemping to understand the case study for the \emph{Well-Typed} interpreter from the \idris{} tutorial \cite[Section~6]{Community2014} some of the concepts presented and encoded require some thought for those not versed in programming language theory.
Even the elementary topics.
The \emph{Well-Typed Interpreter} presents an interpreter for the Well-Typed $\lambda$ calculus.

Fortunatly, I was directed to the excellent blog post \citetitle{Siek2012ccp} by \citeauthor{Siek2012ccp} \cite{Siek2012ccp} by a colleague.
In this blog, \citeauthor{Siek2012ccp} provides the `crash course' through introduction of the \allang{}.
However, these concepts were not necessarily enforced through implementations.
You might as well put theory into practise, no?

\subsection{Contribution}
\label{sec:intro:contribution}

This tutorialss' contributions are summarised as follows:

\paragraph{Contribution 1}
I present my notes on marrying the theory and practise for the \allang{}, illustrating how the theoretical concepts presented in \citet{Siek2012ccp} can be mapped to an interpreter written in \idris{}.

\paragraph{Contribution 2}
I present the \emph{Well-Typed} interpreter from the \idris{} tutorial, together with additional material explaining the underlying formal methods used.

\subsection{Intended Audience}
\label{sec:intro:audience}

This tutorial is presented as my notes\footnote{There will be mistakes, missing information, typos, and \emph{Citations Required}.}.
The intended audience are those who are not familiar with elementary concepts in programming language theory, and have \emph{some} familiarity with \idris{}.
The material is presented in a possible more verbose fashion than required.
This is to be explicit about what is going on.
If you are not familiar with \idris{}, I recommend the \citetitle{Community2014}\cite{Community2014}.


\section{Well-Typed Arithmetic}
\label{sec:typed-arith}

\noindent
This section will introduce some of the basic concepts in programming language theory.
We will consider the construction of a simple language that performs integer arithmetic.
This section will cover:

\begin{compactitem}
\item Grammars
\item Basic Type Theory
  \begin{compactitem}
  \item Types
  \item Simple Typing Contexts
  \item Simple Typing Rules
  \end{compactitem}
\item Denotational Semantics.
\end{compactitem}

\subsection{Modelling Language Syntax}
\label{sec:typed-arith:syntax}

Like natural languages spoken by people today, programming languages have their own syntax and grammar rules.
These rules define the permissible expressions that are to be found within a language.
Fortunatly, programming languages are not as complex (i.e. are context free) as natural language and as such their syntax can be describe rather conscisly.
The most common notation used to present syntax is that of \ac{bnf}.
There are several variants of \ac{bnf} that prove popular, for example: \ac{ebnf} \cite{iso14977} and \ac{abnf} \cite{rfc5234}.

A example of a \ac{bnf} grammar to describe the syntax for a simple arithmetic language is as follows:

\begin{bnf}
\bnfprod{AL}{%
  \bnfts{integer}
}\\
\bnfprod{AL}{%
  \bnfts{``-''}\bnfpn{AL}
}\\
\bnfprod{AL}{%
  \bnfpn{AL}\bnfts{``+''}\bnfpn{AL}
}\\
\bnfprod{AL}{%
  \bnfpn{AL}\bnfts{``-''}\bnfpn{AL}
}\\
\bnfprod{AL}{%
  \bnfpn{AL}\bnfts{``*''}\bnfpn{AL}
}\\
\bnfprod{AL}{%
  \bnfpn{AL}\bnfts{``/''}\bnfpn{AL}
}\\
\end{bnf}

\noindent
This language allows for expressions that describe several binary operations on integers, and a single unary operation.
Notice how expressions are defined in terms of themselves.
Such grammars can be modelled within \idris{} as a simple data type.

\begin{code}
data Arith : -> Type where
  Val : Int            -> Arith
  Neg : Arith          -> Arith
  Add : Arith -> Arith -> Arith
  Sub : Arith -> Arith -> Arith
  Div : Arith -> Arith -> Arith
  Mul : Arith -> Arith -> Arith
\end{code}

\noindent
Here we use data type constructors to model different expressions.
We also embedd an interpretations of integers.
Integers will be mapped to the \idris{} data types.
The syntax presented in the \ac{bnf} grammar can be made more precise through use of an \emph{or} combinator.

\begin{bnf}
\bnfprod{AL}{%
  \bnfts{integer}
  \bnfor
  \bnfts{``-''}\bnfpn{AL}
  \bnfor
  \bnfpn{}ops
}\\
\bnfprod{ops}{%
  \bnfpn{AL}\bnfts{``+''}\bnfpn{AL}
  \bnfor
  \bnfpn{AL}\bnfts{``-''}\bnfpn{AL}
  \bnfor
  \bnfpn{AL}\bnfts{``/''}\bnfpn{AL}
  \bnfor
  \bnfpn{AL}\bnfts{``*''}\bnfpn{AL}
}
\end{bnf}

\noindent
Similar optimisations can be made to the \texttt{Arith} data type.

\begin{code}
data Arith = Val Int
           | Neg Arith
           | Add Arith Arith
           | Sub Arith Arith
           | Div Arith Arith
           | Mul Arith Arith
\end{code}

\noindent
\emph{Programming Language Theorists} (PLTs) are in the business of creating languages and \ac{bnf} (and its popular variants) while descriptive and good to describe implementations can be too verbose.
PLTs use a particular variant of \ac{bnf} in which the name of the language being defined (\allang{}) is replaced by a variable used to range over all the possible values of \allang{}.
The above \ac{bnf} grammer would be re-expressed as follows:

\begin{bnf}
\bnfprod{$e$}{%
  \bnfts{$i$}
  \bnfor
  \bnfts{$-e$}
  \bnfor
  \bnfts{$e+e$}
  \bnfor
  \bnfts{$e-e$}
  \bnfor
  \bnfts{$e/e$}
  \bnfor
  \bnfts{$e*e$}
}
\end{bnf}

\subsection{Basic Types}
\label{sec:typed-arith:types}

\glsplural{grammar} helps us to define the syntax of our language, essentially \emph{what we say}.
\gls{tysys} helps us define \emph{a means to know what we say is correct}.
Types are used to helps us take note of the \emph{kind} of objects that will exist when the program is executed.
These objects will represent the types in our type system.
The set of types in a type system is often represented as $T$.
For the language presented earlier there is only one type of object in play integers $\mathbb{Z}$.

\begin{bnf}
\bnfprod{$T_{Arith}$}{%
  \bnfts{$\mathbb{Z}$}
}
\end{bnf}

\noindent
When modelling types in \idris{} a data \texttt{Ty} type can be constructed.
The type object for the \allang{} is simply:

\begin{code}
data ArithTy = TyValue
\end{code}

\noindent
The purpose of a type system is to allow for the type of an expression to be calculated from the expression itself.
Continuting with the \allang{}, for example, the expression $e=(1 + 2)$ will have type $\mathbb{Z}$ as the result of evaluating $e$ will be $3$ which is an integer.

Type systems are defined using relations that will allow for the pairing of expressions to types.
We call this relation \textsf{WellTyped}, and will contain only: \emph{correctly typed expresions paired with their type}.
Thus:
\[
((1+2), Int)\in\mathsf{WellTyped}\\
(\text{``Bob''},Int)\notin\mathsf{WellTyped}
\]

\noindent
Taking our \idris{} translations \textsf{WellTyped} can be represented as a list of expression type pairs.

\begin{code}
welltyped : List (Arith, ArithTy)
welltyped = [(Add (Val 1) (Val 2), TyValue)]
\end{code}

\noindent
Note we do not have a means (yet) to ensure that only well-typed expressions are constructed.
These are typing rules and are introduced in Section~\ref{sec:typed-arith:rules}.
Before we can specify typing rules, we need to first introduce the notion of \emph{Typing Environments}.

\subsection{Typing Environments}
\label{sec:typed-arith:type-env}

When working with languages keeping track of what elements in the language have what types is important.
For simple languages, such as the one introduced in this section, there is no need: All expressions have the same type.
However, in languages with variables, simple relations are not sufficient.
\emph{Typing Environments} are used to keep track of local variables and their types, and are explained in more depth in the next section.

Traditionally, typing environments are denoted by the greek letter $\Gamma$.
For the purposes of this section, modelling complete typing environments is not required and our typing environment $\Gamma$ just needs to nominally exist.

Using this notion of a type environment we can improve our definition of $\mathsf{WellTyped}$ to include triples in the form of $(\Gamma,e,T)$.
The $\mathsf{WellTyped}$ set will contain expressions that have a type $T$ derived from a local context $\Gamma$.
\[
(\Gamma,e,T)\in\mathsf{WellTyped}
\]
\noindent
To save on typing, the short hand $\Gamma\vdash e:T$ is used.
As our typing environment is empty the empty set symbol is used instead of $\Gamma$.

\subsection{Typing Rules}
\label{sec:typed-arith:rules}

Types and typing environments act as building blocks to help us construct well-typed programs.
To construct the set of relations for \textsf{WellTyped}, we need to define \emph{Typing Rules} that specify how expressions are typed and how types interact when expressions are combined.

\subsubsection{What are Typing Rules?}
\label{sec:typed-arith:rules:what}

Typing rules are a series of judgments that work in a particular context, with the top line defining the inputs and the bottom line the result.
The language in consideration in this section will have the following small set of typing rules.
Integers are given the type $\mathbb{Z}$.

\begin{prooftree}
\AxiomC{}
\LeftLabel{Integers}
\UnaryInfC{$\emptyset\vdash i : \mathbb{Z} $}
\end{prooftree}

\noindent
Negation is only applied to integers.
\begin{prooftree}
\AxiomC{$\emptyset\vdash e : \mathbb{Z}$}
\LeftLabel{Addition}
\UnaryInfC{$\emptyset\vdash -e : \mathbb{Z}$}
\end{prooftree}

\noindent
Operations only work with integers:

\begin{prooftree}
\AxiomC{$\emptyset\vdash e_1 : \mathbb{Z}$}
\AxiomC{$\emptyset\vdash e_2 : \mathbb{Z}$}
\LeftLabel{Addition}
\BinaryInfC{$\emptyset\vdash e_1+e_2 : \mathbb{Z}$}
\end{prooftree}

\begin{prooftree}
\AxiomC{$\emptyset\vdash e_1 : \mathbb{Z}$}
\AxiomC{$\emptyset\vdash e_2 : \mathbb{Z}$}
\LeftLabel{Subtraction}
\BinaryInfC{$\emptyset\vdash e_1-e_2 : \mathbb{Z}$}
\end{prooftree}

\begin{prooftree}
\AxiomC{$\emptyset\vdash e_1 : \mathbb{Z}$}
\AxiomC{$\emptyset\vdash e_2 : \mathbb{Z}$}
\LeftLabel{Multiplication}
\BinaryInfC{$\emptyset\vdash e_1*e_2 : \mathbb{Z}$}
\end{prooftree}

\begin{prooftree}
\AxiomC{$\emptyset\vdash e_1 : \mathbb{Z}$}
\AxiomC{$\emptyset\vdash e_2 : \mathbb{Z}$}
\LeftLabel{Division}
\BinaryInfC{$\emptyset\vdash e_1/e_2 : \mathbb{Z}$}
\end{prooftree}

\subsubsection{Modelling Typing Rules}
\label{sec:typed-arith:rules:modelling}

In non dependently typed languages we can \emph{model} typing rules through pattern matching.
For example here is na\"{i}ve implementation of \texttt{Addition}:
\begin{code}
addition : Arith -> Arith -> Maybe Arith
addition (Value a) (Value b) = Just (Value (a + b))
addition _         _         = Nothing
\end{code}

\noindent
However, there are two problems with this implementation.
First, this might be a bad implementation through the use of \texttt{Maybe}\footnote{
My programming foo is not strong with this.}
Secondly, it is rather verbose, and requires the creation of special functions to construct expressions; we have data constructors for that.

Here is a somewhat better implementation of the typing rules.
We leverage our ability to use dependent types, and embedd the typing rules directly within the constructors of the data type \texttt{Arith}

\begin{code}
data Arith : ArithTy -> Type where
  Value    : Int                            -> Arith TyValue
  Var      : String        -> Arith TyValue -> Arith TyValue
  Neg      : Arith TyValue                  -> Arith TyValue
  Addition : Arith TyValue -> Arith TyValue -> Arith TyValue
\end{code}

\noindent
Just look at that compact representation and emebedding of the typing rules.
Dependent types cool!


\paragraph{Note}
An alternate means to model this simple typed Arithmetic language is to introduce types for the operations.
This will allow for a more compact and stronger definition.
We leave this as an exercise for the reader.

\subsection{Denotational Semantics}
\label{sec:typed-arith:semantics}

So far we have introduced: how to model syntax, represent types, and enforce typing rules.
In this section we introduce the notion of \emph{Denotational Semantics} to provide an \emph{interpretation} of the language into \idris{}.
Denotational semantics is a technique that allows us to define the semantics of a language's expressions using another base notation, usually set-theory.
However, \idris{} is essentially applied maths\ldots and I am too lazy to provide a full denotational semantics for the language\footnote{This is left as an exercise for the author\ldots}.
Instead  we will provide an interpretation of in \idris{} such that we can build an interpreter that will allow us to interpret and execute programs.
Here the notation $\interpB{e}$ will be used to denote the interpretation of an element.

Alternatively, the definition of a language's semantics can be defined computationally using operational semantics\footnote{\textsc{Ibid}}.

\subsubsection{Interpreting Types}
\label{sec:typed-arith:semantics:types}

We begin by providing an interpretation of the types $T_{Arith}$:

\begin{center}
\begin{tabularx}{0.8\textwidth}{>{$}r<{$}>{\ttfamily}X}
\interpB{\mathbb{Z}}=& Int \\
\end{tabularx}
\end{center}

\noindent
A type interpreter can be written as follows:

\begin{code}
interpTy : ArithTy -> Type
interpTy TyValue = Int
\end{code}

\subsubsection{Expressions}
\label{sec:typed-arith:semantics:exrs}

Now we look to interpret expressions.

\begin{center}
\begin{tabularx}{0.8\textwidth}{>{$}r<{$}>{\ttfamily}X}
\interpB{i}     =& i\\
\interpB{-e}    =& (-1) * $\interpB{e}$\\
\interpB{x + y} =& $\interpB{x}$ + $\interpB{x}$ \\
\interpB{x - y} =& $\interpB{x}$ - $\interpB{x}$ \\
\interpB{x / y} =& $\interpB{x}$ `div` $\interpB{x}$ \\
\interpB{x * y} =& $\interpB{x}$ * $\interpB{x}$ \\
\end{tabularx}
\end{center}

\noindent
Raw values are directly translated into \idris{} values with type \texttt{Int}.
Negative numbers are interpreted expressions multiplied by $-1$.
Finally, the binary operations are mapped directly to their \idris{} equivalents.
Our interpreter for the language is thus constructed as follows:

\begin{code}
interp : Arith t -> interpTy t
interp (Val x)   = x
interp (Neg x)   = (-1) * (interp x)
interp (Add x y) = (interp x) + (interp y)
interp (Sub x y) = (interp x) - (interp y)
interp (Div x y) = (interp x) `div` (interp y)
interp (Mul x y) = (interp x) * (interp y)
\end{code}

\noindent
Not the similarities between the formalised semantics and representation in \idris{}.

\subsection{Running the Interpreter.}
\label{sec:running-interpreter}

Now that we have constructed the interpreter for the language we now can use the \texttt{interp} function to execute expressions.

\begin{code}
main : IO ()
main = do
  let expr = (Add (Val 1) (Val 2))
  print $ interp expr
\end{code}

%%% Local Variables:
%%% mode: latex
%%% TeX-master: "../tutorial.print"
%%% End:

\section{Well-Typed Arithmetic with Variables}
\label{sec:typed-arith-var}

In this section we will extend our well-typed arithmetic language to include the use of free variables.
Specifically, this section will provide further coverage on:
\begin{compactitem}
  \item Typing Contexts; and
  \item Typing Rules.
\end{compactitem}

\subsection{Language Syntax}
\label{sec:typed-arith-var:syntax}

We extend the syntax of our language to include variables:

\begin{bnf}
\bnfprod{$e$}{%
  \bnfts{$i$}
  \bnfor
  \bnfts{$x$}
  \bnfor
  \bnfts{$-e$}
  \bnfor
  \bnfts{$e+e$}
  \bnfor
  \bnfts{$e-e$}
  \bnfor
  \bnfts{$e/e$}
  \bnfor
  \bnfts{$e*e$}
}
\end{bnf}

\noindent
Variables will be represented as $x$.
Before the resulting \idris{} representation can be given we must first re-examine our notion of typing environments and introduce typing contexts.

\subsection{Types}
\label{sec:typed-arith-var:types}

When presenting the typing environment in Section~\ref{sec:typed-arith:type-env} the language did not have variables and as such a nominal ``empty'' typing environment was sufficient.
However, we have now introduced variables.
Working with variables is problematic in that variables are  mutable and the type of a variable will not be known \emph{a priori}, and may change during program execution\footnote{Well with this particular language, variables will always have the same type.}.
To address this issue we partially introduced the concept of a \emph{typing environment} in Section~\ref{sec:typed-arith:type-env}.

Traditionally, typing environments are denoted by the greek letter $\Gamma$.
A typing environment is defined as a set that contains tuples of variable values and their types: $(x,T)$.
Type environments can be extended, and the addition of new variables to the environment may result in old definitions being removed.

When working with typing environments it is important to consider the list of available types separately from their values.
This separation allows us to consider the types for particular contexts outside what their values actually are.

\subsubsection{Typing Contexts}
\label{sec:typed-arith-var:contexts}

We define a \emph{typing context} as the pairing between a representation of a variable and its type.
We first construct a generic type alias \texttt{Context ty}, that represents a typing context with type \texttt{ty}.

\begin{code}
Context : Type -> Type
Context ty = List (String, ty)
\end{code}

\noindent
Using this alias we can then create typing environments of type \texttt{Context ty}.
For example type environments using our language can be modelled as

\begin{code}
example_context : Context ArithTy
example_context = [("foo", TyValue), ("bar", TyValue)]
\end{code}

\noindent
\textbf{Note} Here we are purposefully using a \emph{named representation} and keeping track of variables using their actual name.
This makes it easier to understand what is going on within contexts.
An alternate would be to use a \emph{nameless representation}, in our larger case study (Section~\ref{}) we will show how a nameless representation using \emph{de Bruijn} indices works.

Before type environments can be used in anger we define several operations that operate on $\Gamma$.

\paragraph{Removing}
\label{sec:typed-arith-var:types:remove}

The first operation will update $\Gamma$ by removing all references to $x$.
\[
\Gamma\backslash x
\]
\noindent
This operation can be implemented as follows:
\begin{code}
remove : (String, ty) -> Context ty -> Context ty
remove e es = deleteBy (\(x,y),(c,d) => x==c) e es
\end{code}

\paragraph{Extending}
\label{sec:typed-arith-var:types:extend}

The second operations is used to extend $\Gamma$ with a variable $x$ of type $T$.
This operation will overide previous $x$.
\[
\Gamma,x:T = (\Gamma\backslash x)\cup\{x,T\}
\]
\noindent
Which can be representened as:
\begin{code}
extend : (String, ty) -> Context ty -> Context ty
extend e es = e :: (remove e es)
\end{code}

\paragraph{Searching}
\label{sec:typed-arith-var:types:lookup}

For completness, the final operation defined is used to search the typying environment to extract the type for a specified variable.
\[
\mathsf{lookup}(\Gamma,x:T) = \text{??}
\]
\noindent
This can be implemented as:
\begin{code}
lookup : String -> Context ty -> Maybe ty
lookup = List.lookup
\end{code}

\paragraph{Improving \idris{} Implementation}
\label{sec:typed-arith-var:types:classes}

We can enforce our formal models of typing contexts in \idris{} better by creating a type class that groups our default implementations together as follows:

\begin{code}
class (Eq ty) => TypingContext ty where
  remove : (String, ty) -> Context ty -> Context ty
  extend : (String, ty) -> Context ty -> Context ty
  lookup : String       -> Context ty -> Maybe ty

  remove e es = deleteBy (\(x,y),(c,d) => x==c) e es
  extend e es = e :: (remove e es)
  lookup = List.lookup
\end{code}

\paragraph{Examples}
\label{sec:typed-arith-var:types:examples}

Explaining changes to a typing context is difficult to observe with type systems that only have a single type.
We further motivate the need for typing contexts by using an alternate type system in which we introduce an additional type: $\mathbb{B}$ for expressions of type boolean.
Such boolean types will have a type constructor $TyBool$.
Examples of typing contexts and their operations are as follows:

\[
\Gamma_{1}=\{(foo,\mathbb{Z}),(g,\mathbb{B}\}
\]

\begin{code}
ctxt : Context ArithTy
ctxt = [("foo", TyValue), ("g", TyBool)]
\end{code}

\noindent
$\Gamma_{1}$ can be updated as follows:
\[
\Gamma_{1},g:\mathbb{Z} =\{(foo,\mathbb{Z}),(g,\mathbb{Z})\}
\]
\begin{code}
env' : TypingContext ArithTy => Context ArithTy
env' = extend ("g", TyValue) ctxt
\end{code}
\noindent
and:
\[
\Gamma_{1},f:\mathbb{B}=\{(foo,\mathbb{Z}),(g,\mathbb{Z}),(f,\mathbb{B})\}
\]
\begin{code}
env'' : TypingContext ArithTy => Context ArithTy
env'' = extend ("f", TyBool) env'
\end{code}

\subsubsection{Typing Environments }
\label{sec:typed-arith-var:environments}

Now that we have defined a typing context we look to define a typing environment.
Recall that our typing environment is defined a set of tuples containing variable values and their types: $(x,T)$.
This can be represented in \idris{} as a variant of a list in which the data type contains the typing context:

\begin{code}
data Env : Context ArithTy -> Type where
  Nil  : Env Nil
  (::) : (e : (String, Int)) -> Env g
       -> Env (extend (fst e, TyValue) g)
\end{code}

\noindent
We also construct a complementary function that allows us to extract the value of a variable from the environment.

\begin{code}
getValue : String -> Env g -> Int
getValue n Nil           = 0
getValue n ((a,v) :: xs) = case n == a of
  True => v
  False => fromEnv n xs
\end{code}

\subsection{Additional Typing Rules}
\label{sec:typed-arith-var:typing-rules}

With our new knowledge of typing environments we can now update our list of typing rules to include variables:

\begin{prooftree}
\AxiomC{$(x,T)\in\Gamma$}
\AxiomC{$T:\mathbb{Z}$}
\LeftLabel{Variables}
\BinaryInfC{$\Gamma\vdash i : \mathbb{Z}$}
\end{prooftree}

\noindent
In our language, a variable will be associated with a type $T$ if that variable exists within the typing environment.
That is if the result of a look-up in the typing context returns the type and not \texttt{Nothing}.
For example:

\begin{code}
ctxt : Context ArithTy
ctxt = [("foo", TyValue), ("g", TyBool)]

test : Bool
test = (Just TyValue) == lookup "foo" ctxt
\end{code}

\noindent
However, as our language only has a single type we can skip this step\footnote{Or maybe not\ldots}.

\subsection{New Model}
\label{sec:typed-arith-var:model}

Armed with our knowledge of typing contexts and extended set of typing rules, we can now proceed to define new version of the \texttt{Arith} data type in which we embed the typing context $\Gamma$ within each expression.

\begin{code}
data Arith : Context ArithTy -> ArithTy -> Type where
  Val : Int                                -> Arith g TyValue
  Var : (n : String)                       -> Arith g TyValue
  Neg : Arith g TyValue                    -> Arith g TyValue
  Add : Arith g TyValue -> Arith g TyValue -> Arith g TyValue
  Sub : Arith g TyValue -> Arith g TyValue -> Arith g TyValue
  Div : Arith g TyValue -> Arith g TyValue -> Arith g TyValue
  Mul : Arith g TyValue -> Arith g TyValue -> Arith g TyValue
\end{code}

\noindent
And redefine our interpreter to work with new definition of \texttt{Arith}.

\begin{code}
interp : Env g -> Arith g TyValue -> Int
interp env (Val x)    = x
interp env (Var n)    = getValue n env
interp env (Neg x)    = (-1) * (interp env x)
interp env (Add x y)  = (interp env x) + (interp env y)
interp env (Sub x y)  = (interp env x) - (interp env y)
interp env (Div x y)  = (interp env x) `div` (interp env y)
interp env (Mul x y)  = (interp env x) * (interp env y)
\end{code}

\subsection{Executing Programs}
\label{sec:typed-arith-var:running}

We can now proceed to execute programs.
As our programs use free variables, variables and their values must be defined within the environment.
For example we define the program: $1 + x$, and then execute the program with $x=3$.

\begin{code}
main : IO ()
main = do
  print $ interp [("x",3)] (Add (Val 1) (Var "x"))
\end{code}

%%% Local Variables:
%%% mode: latex
%%% TeX-master: "../tutorial.print"
%%% End:

%\section{Well-Typed $\lambda$-Calculus}
\label{sec:lambda}


This section will present a more comprehensive example that utilises the concepts seen in this tutorial.
The example will be a variant of the well-typed $\lambda$ calculus.
Specifically, this section will provide further coverage of:
\begin{compactitem}
  \item Typing Contexts; and
  \item Typing Rules.
\end{compactitem}
We will do so through consideration of nameless representations for typing environments and contexts.

\subsection{Language Syntax}
\label{sec:lambda:syntax}

We begin by introducing the syntax for our variant of the $\lambda$-calculus.
This variant of the language, which we refer to as \lamlang{}, was originally introduced in \citet{Community2014}.
We present a more specific variant in which a predefined set of binary operations are to be modelled.

\textbf{Note} the variant presented in this section does not have an \emph{as} compact representation as that presented in \citet{Community2014}.
We use a more obtuse representation on purpose.

\lamlang{} will support the following constructs:
\begin{inparaenum}
  \item Values,
  \item Variables,
  \item Binary Operators, representing arithmetic, boolean, and relational oeprations,
  \item Function application,
  \item Conditional statements; and
  \item Anonymous functions.
\end{inparaenum}
The grammar rules are:

\begin{bnf}
\bnfprod{$e$}{%
  \bnfts{$i$}
  \bnfor
  \bnfts{$x$}
  \bnfor
  \bnfts{$e\circ e$}
  \bnfor
  \bnfts{$e\;\$\;e$}
  \bnfor
  \bnfts{$e\;?\;e : e$}
  \bnfor
  \bnfts{$\lambda x:T.e$}
}\\
\bnfprod{$\circ$}{
  \bnfts{$\circ_{a}$}
  \bnfor
  \bnfts{$\circ_{r}$}
  \bnfor
  \bnfts{$\circ_{b}$}
}\\
\bnfprod{$\circ_{a}$}{
  \bnfts{$/$}
  \bnfor
  \bnfts{$*$}
  \bnfor
  \bnfts{$+$}
  \bnfor
  \bnfts{$-$}
}\\
\bnfprod{$\circ_{r}$}{
  \bnfts{$>$}
  \bnfor
  \bnfts{$<$}
  \bnfor
  \bnfts{$\equiv$}
}\\
\bnfprod{$\circ_{b}$}{
  \bnfts{$\wedge$}
  \bnfor
  \bnfts{$\vee$}
}
\end{bnf}

\noindent
In \lamlang{}:

\begin{center}
\begin{tabular}[h]{>{$}l<{$}l}
i       & are integer variables. \\
x       & represents variable names.\\
\circ   & is an infix binary operation.\\
\$      & is function application.\\
?       & is a ternary representation of a conditional statement.\\
\lambda & are anonymous functions in which the function argument $x$ has type $T$.\\
\end{tabular}
\end{center}

\noindent
The syntax for \lamlang{} can be na\"{i}vly modelled in \idris{} as follows:

\begin{code}
data Expr = Val Int
          | Var String Expr
          | Div Expr Expr | Mul Expr
          | Add Expr Expr | Sub Expr Expr
          | GT Expr Expr  | LT Expr Expr | Eqv Expr Expr
          | And Expr Expr | OR Expr Expr
          | App Expr Expr
          | If Expr Expr Expr
          | Lam Expr
\end{code}

\noindent
However, a smarter representation of \lamlang{} is to generalise binary operations according to their type such that an operation is defined as a function that is applied to two expressions.
However, a even more compact representation is to have a single generic binary expression operation.
\begin{code}
data Expr = Val Int
          | Var String Expr
          | Op (Expr -> Expr -> Expr) Expr Expr
          | App Expr Expr
          | If Expr Expr Expr
          | Lam Expr
\end{code}

\noindent
Example instances of programs written in \lamlang{} include:

\begin{align*}
  1 + 2\\
  3 > 4\\
  \backslash x.\backslash y. x + y\\
 \backslash x.(\neg x)\\
\end{align*}



\subsection{Types Revisited}
\label{sec:lambda:types}

Within \lamlang{} the types of each value will be either integers $\mathbb{Z}$, booleans $\mathbb{B}$, and functions and will be defined as transformation beteen two types.

\begin{bnf}
\bnfprod{$T$}{%
  \bnfts{$\mathbb{Z}$}
  \bnfor
  \bnfts{$\mathbb{B}$}
  \bnfor
  \bnfts{$T\rightarrow T$}
}
\end{bnf}

A function type indicates the mapping from the type of the arugment to the type of the return value.
The purpose of a type system is to allow for the type of an expression to be calculated from the expression itself.
For example, the expression $e=(1 + 2)$ will have type $\mathbb{Z}$ as the result of evaluating $e$ will be $3$ which is an integer.

These types can be modelled in \idris{} as follows:

\begin{code}
data Ty = TyValue
        | TyBool
        | TyFunc Ty Ty
\end{code}

\noindent
Again it is interesting to note how we can model in \idris{} types and grammars in similar ways.

\subsection{Types Environments Revisited}
\label{sec:lambda:typing}

When presenting the typing environment in Section~\ref{sec:typed-arith:type-env} the language did not have variables and as such a nominal ``empty'' typing environment was sufficient.
Section~\ref{sec:typed-arith-var:environments} introduced how typing environments  and contexts can be used to keep track of variables, their values, and their types.
This was illustrated using a \emph{named representation}.
In this section we will use a \emph{nameless representation} called \emph{de Bruijn indices} as an alternate representation.
\todo{Add math definition for de Bruijn incidies}.

\subsubsection{Contexts}
\label{sec:lambda:typing:contexts}

Contexts can be simply implemented as a vector of types.
As the context is to be used regularly as an implicit argument, we can define all the implementation within a \texttt{using} block.

\begin{code}
using (g:Vect n Ty)
\end{code}

\subsubsection{Membership Proof}
\label{sec:lambda:typing:membership-proof}

\emph{de Bruijn indicies} are a nameless representation in which variables are represented by a proof that states their membership of a particular context.
This proof can be represented using the data type \texttt{HasType i G T}.
Which reads a proof that variable \texttt{i} in context \texttt{G} has type \texttt{T}.
We can define it as follows:

\begin{code}
data HasType : (i : Fin n) -> Vect n Ty -> Ty -> Type where
    Recent : HasType FZ (t :: g) t
    Next   : HasType k g t -> HasType (FS k) (u :: g) t
\end{code}

\noindent
\texttt{HasType} is an inductive style proof.
The constructor \texttt{Recent} can be seen as the base case which states that the most recently defined variable is well typed.
The constructor \texttt{Next n} is the inductive case that states if the $n^{th}$ most recently defined variable is well-typed, then so is the $n+1^{th}$ variable.
With this \emph{proof}, \texttt{Recent} represents the most recently defined variables, and \texttt{Next Recent} to refer to the next variable and so on.

\subsubsection{Environments}
\label{sec:lambda:typying:envs}

To keep track of the types in scope, we need an environment.
Our definition of environment changes little from Section~\ref{sec:typed-arith-var:environments}.
We drop the variable name value pairings and just store the values.

\begin{code}
data Env : Vect n Ty -> Type where
    Nil  : Env Nil
    (::) : interpTy a -> Env g -> Env (a :: g)
\end{code}

\noindent
As with our definition of the context, we store the most recently defined variable at the head of the list.
We also abuse the list construct so that the list syntax from \idris{} can be used.

We can also define a function \texttt{lookup} that when given a proof that a variable is defined within the context, a value can be extracted from the environment.

\begin{code}

lookup : HasType i g t -> Env g -> interpTy t
lookup Recent   (x :: xs) = x
lookup (Next k) (x :: xs) = lookup k xs
\end{code}

\subsection{Typing Rules}
\label{sec:lambda:rules}

Here we define the rules that allow us to construct \emph{well-typed} expressions.

\subsubsection{Values}
\label{sec:type:rules:values}
Here we define integer and boolean values and assign them a type.
\begin{prooftree}
\AxiomC{}
\LeftLabel{Integers}
\UnaryInfC{$\Gamma\vdash i : \mathbb{Z} $}
\end{prooftree}

\begin{prooftree}
\AxiomC{}
\LeftLabel{Booleans}
\UnaryInfC{$\Gamma\vdash b : \mathbb{B} $}
\end{prooftree}

\subsubsection{Arithmetic Operations}
\label{sec:type:rules:arith}

Arithmetic operations have the following types.

\begin{prooftree}
\AxiomC{$\Gamma\vdash e_1 : \mathbb{Z}$}
\AxiomC{$\Gamma\vdash e_2 : \mathbb{Z}$}
\LeftLabel{Division}
\BinaryInfC{$\Gamma\vdash e_1/e_2 : \mathbb{Z}$}
\end{prooftree}

\begin{prooftree}
\AxiomC{$\Gamma\vdash e_1 : \mathbb{Z}$}
\AxiomC{$\Gamma\vdash e_2 : \mathbb{Z}$}
\LeftLabel{Multiplication}
\BinaryInfC{$\Gamma\vdash e_1*e_2 : \mathbb{Z}$}
\end{prooftree}

\begin{prooftree}
\AxiomC{$\Gamma\vdash e_1 : \mathbb{Z}$}
\AxiomC{$\Gamma\vdash e_2 : \mathbb{Z}$}
\LeftLabel{Addition}
\BinaryInfC{$\Gamma\vdash e_1+e_2 : \mathbb{Z}$}
\end{prooftree}

\begin{prooftree}
\AxiomC{$\Gamma\vdash e_1 : \mathbb{Z}$}
\AxiomC{$\Gamma\vdash e_2 : \mathbb{Z}$}
\LeftLabel{Subtraction}
\BinaryInfC{$\Gamma\vdash e_1-e_2 : \mathbb{Z}$}
\end{prooftree}

\subsubsection{Boolean Operations}
\label{sec:types:rules:boolean}

Boolean operations can only work on bools.

\begin{prooftree}
\AxiomC{$\Gamma\vdash e_1 : \mathbb{B}$}
\AxiomC{$\Gamma\vdash e_2 : \mathbb{B}$}
\LeftLabel{And}
\BinaryInfC{$\Gamma\vdash e_1\wedge e_2 : \mathbb{B}$}
\end{prooftree}

\begin{prooftree}
\AxiomC{$\Gamma\vdash e_1 : \mathbb{B}$}
\AxiomC{$\Gamma\vdash e_2 : \mathbb{B}$}
\LeftLabel{Or}
\BinaryInfC{$\Gamma\vdash e_1\vee e_2 : \mathbb{B}$}
\end{prooftree}

\begin{prooftree}
\AxiomC{$\Gamma\vdash e : \mathbb{B}$}
\LeftLabel{Negation}
\UnaryInfC{$\Gamma\vdash \neg e : \mathbb{B}$}
\end{prooftree}

\subsubsection{Relational Operations}
\label{sec:types:rules:relation}

Numerical comparison operations result in boolean results.

\begin{prooftree}
\AxiomC{$\Gamma\vdash e_1 : \mathbb{Z}$}
\AxiomC{$\Gamma\vdash e_2 : \mathbb{Z}$}
\LeftLabel{Greater Than}
\BinaryInfC{$\Gamma\vdash e_1 > e_2 : \mathbb{B}$}
\end{prooftree}

\begin{prooftree}
\AxiomC{$\Gamma\vdash e_1 : \mathbb{Z}$}
\AxiomC{$\Gamma\vdash e_2 : \mathbb{Z}$}
\LeftLabel{Less than}
\BinaryInfC{$\Gamma\vdash e_1< e_2 : \mathbb{B}$}
\end{prooftree}

\begin{prooftree}
\AxiomC{$\Gamma\vdash e_1 : \mathbb{Z}$}
\AxiomC{$\Gamma\vdash e_2 : \mathbb{Z}$}
\LeftLabel{Equality}
\BinaryInfC{$\Gamma\vdash e_1\equiv e_2 : \mathbb{B}$}
\end{prooftree}


\subsubsection{Variables}
\label{sec:types:rules:variables}
Variables can either be booleans or integers.
\begin{prooftree}
\AxiomC{$(x,T)\in\Gamma$}
\AxiomC{$T:\mathbb{Z}$}
\LeftLabel{Integer Variables}
\BinaryInfC{$\Gamma\vdash i : \mathbb{Z} $}
\end{prooftree}

\begin{prooftree}
\AxiomC{$(x,T)\in\Gamma$}
\AxiomC{$T:\mathbb{B}$}
\LeftLabel{Boolean Variables}
\BinaryInfC{$\Gamma\vdash b : \mathbb{B} $}
\end{prooftree}

\subsubsection{Functions}
\label{sec:types:rules:functions}

\begin{prooftree}
\AxiomC{$\Gamma\vdash e_1 : \mathbb{B}$}
\AxiomC{$\Gamma\vdash e_2 : T$}
\AxiomC{$\Gamma\vdash e_3 : T$}
\LeftLabel{Conditionals}
\TrinaryInfC{$\Gamma\vdash (e_1?e_2:e_3): T$}
\end{prooftree}

\begin{prooftree}
\AxiomC{$\Gamma,x:T_1\vdash e : T_2$}
\LeftLabel{Anonymous Functions}
\UnaryInfC{$\Gamma\vdash (\lambda x:T_1.e) : (T_1\rightarrow T_2)$}
\end{prooftree}

\begin{prooftree}
\AxiomC{$\Gamma\vdash e_1 : T\rightarrow\alt{T}$}
\AxiomC{$\Gamma\vdash e_2 : T$}
\LeftLabel{Application}
\BinaryInfC{$\Gamma\vdash e_1 \$ e_2 : \alt{T}$}
\end{prooftree}

\subsection{Model}
\label{sec:lambda:model}
\begin{code}
data Expr : Vect n Ty -> Ty -> Type where
    Var : HasType i g t   -> Expr g t
    Val : (x : Int)       -> Expr g TyInt
    Lam : Expr (a :: g) t -> Expr g (TyFun a t)
    App : Expr g (TyFun a t) -> Expr g a -> Expr g t
    Op  : (interpTy a -> interpTy b -> interpTy c)
            -> Expr g a
            -> Expr g b
            -> Expr g c
    If  : Expr g TyBool
          -> Lazy (Expr g a)
          -> Lazy (Expr g a)
          -> Expr g a
\end{code}

\subsection{Denotational Semantics}
\label{sec:lambda:semantics}

The purpose of this section is to provide an \emph{interpretation} of \lamlang{} to \idris{}.
The next section will illustrate how an interpreter can be constructed in \idris{} to run our \lamlang{} programs.

\subsubsection{Values \& Variables}
\label{sec:semantics:variables}
Values are translated directly into their \idris{} representation.
Variables are special\ldots

\begin{center}
\begin{tabularx}{0.8\textwidth}{>{$}r<{$}>{\ttfamily}X}
\interpB{i}     =& \\
\interpB{x}     =& \\
\end{tabularx}
\end{center}

\subsubsection{Types}
\label{sec:semantics:types}

Types are mapped to their \idris{} equivalents.

\begin{center}
\begin{tabularx}{0.8\textwidth}{>{$}r<{$}>{\ttfamily}X}
\interpB{}     =& \\
\end{tabularx}
\end{center}

\subsection{Binary Operators}
\label{sec:semantics:binary-operations}

All binary operations are mapped directly to their \idris{} versions.

\begin{center}
\begin{tabularx}{0.8\textwidth}{>{$}r<{$}>{\ttfamily}X}
\interpB{}     =& \\
\end{tabularx}
\end{center}

\subsection{Functions \& Conditionals}
\label{sec:semantics:functions}

Conditionals are mapped to the \idris{} ifthenelse construct, and anonymous functions to \idris{}'s own anonymous functions.

\begin{center}
\begin{tabularx}{0.8\textwidth}{>{$}r<{$}>{\ttfamily}X}
\interpB{}     =& \\
\end{tabularx}
\end{center}

%%% Local Variables:
%%% mode: latex
%%% TeX-master: "../tutorial.screen"
%%% End:


\newpage
\printbibliography{}
\newpage
\glsaddall
\printglossary[style=listdotted]
\end{document}
