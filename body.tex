
\input{conf.ltx}

\newcommand{\version}{\gitVtagn}

\title{Introduction to some \emph{Formal Methods} using \idris{}}
\author{Jan de Muijnck-Hughes}
\date{\origdate\today}
\addbibresource{biblio.bib}
\loadglsentries[type=\acronymtype]{./acronyms.gloss}
\loadglsentries{./glossary.gloss}
\makeglossaries

\begin{document}

\maketitle%
\begin{abstract}
This tutorial will provide an introduction to some elementary topics in programming language theory.

\begin{itemize}
\item Namely: grammars, type systems, and denotational semantics.
\item We will do this through construction of interpreters for two simple programming languages: \allang{}, and the well-typed lambda calculus---\lamlang{}.
The dependently typed programming language \idris{} will be used to introduce these concepts \cite{}.
\begin{itemize}
\item \allang{} was orginally described formally in a blog post entitled \citetitle{Siek2012ccp} \cite{Siek2012ccp}.
Here we present the formalisations together with an interpreter written in \idris{}.

\item The interpreter for the well-typed $\lambda$ calculus was originally presented in the offical tutorial for \idris{} \cite{Community2014}.
Here we present the interpreter with some additional material explaining the underlying formal methods.
\end{itemize}
\item More information over Idris can be found online at: \url{http://www.idris-lang.org}
\end{itemize}
\end{abstract}

\newpage
\tableofcontents
\newpage

\section{Introduction}
\label{sec:intro}

At the \emph{original} time of writing this tutorial I was relearning/learning elementary concepts in \emph{Programming Language Theory} that went well-beyond grammars.
Notably: I was looking into topics such as type systems, denotational semantics, and operational semantics.
Given my interest in practical uses of dependent types, in which I use the \idris{} programming language for my research, I decided to code-up these theoretical knowns as \idris{} code.
Moreover, when attemping to understand the case study for the \emph{Well-Typed} interpreter from the \idris{} tutorial \cite[Section~6]{Community2014} some of the concepts presented and encoded require some thought for those not versed in programming language theory.
Even the elementary topics.
The \emph{Well-Typed Interpreter} presents an interpreter for the Well-Typed $\lambda$ calculus.

Fortunatly, I was directed to the excellent blog post \citetitle{Siek2012ccp} by \citeauthor{Siek2012ccp} \cite{Siek2012ccp} by a colleague.
In this blog, \citeauthor{Siek2012ccp} provides the `crash course' through introduction of the \allang{}.
However, these concepts were not necessarily enforced through implementations.
You might as well put theory into practise, no?

\subsection{Contribution}
\label{sec:intro:contribution}

This tutorialss' contributions are summarised as follows:

\paragraph{Contribution 1}
I present my notes on marrying the theory and practise for the \allang{}, illustrating how the theoretical concepts presented in \citet{Siek2012ccp} can be mapped to an interpreter written in \idris{}.

\paragraph{Contribution 2}
I present the \emph{Well-Typed} interpreter from the \idris{} tutorial, together with additional material explaining the underlying formal methods used.

\subsection{Intended Audience}
\label{sec:intro:audience}

This tutorial is presented as my notes\footnote{There will be mistakes, missing information, typos, and \emph{Citations Required}.}.
The intended audience are those who are not familiar with elementary concepts in programming language theory, and have \emph{some} familiarity with \idris{}.
The material is presented in a possible more verbose fashion than required.
This is to be explicit about what is going on.
If you are not familiar with \idris{}, I recommend the \citetitle{Community2014}\cite{Community2014}.


\section{Well-Typed Arithmetic}
\label{sec:typed-arith}

\noindent
This section will introduce some of the basic concepts in programming language theory.
We will consider the construction of a simple language that performs integer arithmetic.
This section will cover:

\begin{compactitem}
\item Grammars
\item Basic Type Theory
  \begin{compactitem}
  \item Types
  \item Simple Typing Contexts
  \item Simple Typing Rules
  \end{compactitem}
\item Denotational Semantics.
\end{compactitem}

\subsection{Modelling Language Syntax}
\label{sec:typed-arith:syntax}

Like natural languages spoken by people today, programming languages have their own syntax and grammar rules.
These rules define the permissible expressions that are to be found within a language.
Fortunatly, programming languages are not as complex (i.e. are context free) as natural language and as such their syntax can be describe rather conscisly.
The most common notation used to present syntax is that of \ac{bnf}.
There are several variants of \ac{bnf} that prove popular, for example: \ac{ebnf} \cite{iso14977} and \ac{abnf} \cite{rfc5234}.

A example of a \ac{bnf} grammar to describe the syntax for a simple arithmetic language is as follows:

\begin{bnf}
\bnfprod{AL}{%
  \bnfts{integer}
}\\
\bnfprod{AL}{%
  \bnfts{``-''}\bnfpn{AL}
}\\
\bnfprod{AL}{%
  \bnfpn{AL}\bnfts{``+''}\bnfpn{AL}
}\\
\bnfprod{AL}{%
  \bnfpn{AL}\bnfts{``-''}\bnfpn{AL}
}\\
\bnfprod{AL}{%
  \bnfpn{AL}\bnfts{``*''}\bnfpn{AL}
}\\
\bnfprod{AL}{%
  \bnfpn{AL}\bnfts{``/''}\bnfpn{AL}
}\\
\end{bnf}

\noindent
This language allows for expressions that describe several binary operations on integers, and a single unary operation.
Notice how expressions are defined in terms of themselves.
Such grammars can be modelled within \idris{} as a simple data type.

\begin{code}
data Arith : -> Type where
  Val : Int            -> Arith
  Neg : Arith          -> Arith
  Add : Arith -> Arith -> Arith
  Sub : Arith -> Arith -> Arith
  Div : Arith -> Arith -> Arith
  Mul : Arith -> Arith -> Arith
\end{code}

\noindent
Here we use data type constructors to model different expressions.
We also embedd an interpretations of integers.
Integers will be mapped to the \idris{} data types.
The syntax presented in the \ac{bnf} grammar can be made more precise through use of an \emph{or} combinator.

\begin{bnf}
\bnfprod{AL}{%
  \bnfts{integer}
  \bnfor
  \bnfts{``-''}\bnfpn{AL}
  \bnfor
  \bnfpn{}ops
}\\
\bnfprod{ops}{%
  \bnfpn{AL}\bnfts{``+''}\bnfpn{AL}
  \bnfor
  \bnfpn{AL}\bnfts{``-''}\bnfpn{AL}
  \bnfor
  \bnfpn{AL}\bnfts{``/''}\bnfpn{AL}
  \bnfor
  \bnfpn{AL}\bnfts{``*''}\bnfpn{AL}
}
\end{bnf}

\noindent
Similar optimisations can be made to the \texttt{Arith} data type.

\begin{code}
data Arith = Val Int
           | Neg Arith
           | Add Arith Arith
           | Sub Arith Arith
           | Div Arith Arith
           | Mul Arith Arith
\end{code}

\noindent
\emph{Programming Language Theorists} (PLTs) are in the business of creating languages and \ac{bnf} (and its popular variants) while descriptive and good to describe implementations can be too verbose.
PLTs use a particular variant of \ac{bnf} in which the name of the language being defined (\allang{}) is replaced by a variable used to range over all the possible values of \allang{}.
The above \ac{bnf} grammer would be re-expressed as follows:

\begin{bnf}
\bnfprod{$e$}{%
  \bnfts{$i$}
  \bnfor
  \bnfts{$-e$}
  \bnfor
  \bnfts{$e+e$}
  \bnfor
  \bnfts{$e-e$}
  \bnfor
  \bnfts{$e/e$}
  \bnfor
  \bnfts{$e*e$}
}
\end{bnf}

\subsection{Basic Types}
\label{sec:typed-arith:types}

\glsplural{grammar} helps us to define the syntax of our language, essentially \emph{what we say}.
\gls{tysys} helps us define \emph{a means to know what we say is correct}.
Types are used to helps us take note of the \emph{kind} of objects that will exist when the program is executed.
These objects will represent the types in our type system.
The set of types in a type system is often represented as $T$.
For the language presented earlier there is only one type of object in play integers $\mathbb{Z}$.

\begin{bnf}
\bnfprod{$T_{Arith}$}{%
  \bnfts{$\mathbb{Z}$}
}
\end{bnf}

\noindent
When modelling types in \idris{} a data \texttt{Ty} type can be constructed.
The type object for the \allang{} is simply:

\begin{code}
data ArithTy = TyValue
\end{code}

\noindent
The purpose of a type system is to allow for the type of an expression to be calculated from the expression itself.
Continuting with the \allang{}, for example, the expression $e=(1 + 2)$ will have type $\mathbb{Z}$ as the result of evaluating $e$ will be $3$ which is an integer.

Type systems are defined using relations that will allow for the pairing of expressions to types.
We call this relation \textsf{WellTyped}, and will contain only: \emph{correctly typed expresions paired with their type}.
Thus:
\[
((1+2), Int)\in\mathsf{WellTyped}\\
(\text{``Bob''},Int)\notin\mathsf{WellTyped}
\]

\noindent
Taking our \idris{} translations \textsf{WellTyped} can be represented as a list of expression type pairs.

\begin{code}
welltyped : List (Arith, ArithTy)
welltyped = [(Add (Val 1) (Val 2), TyValue)]
\end{code}

\noindent
Note we do not have a means (yet) to ensure that only well-typed expressions are constructed.
These are typing rules and are introduced in Section~\ref{sec:typed-arith:rules}.
Before we can specify typing rules, we need to first introduce the notion of \emph{Typing Environments}.

\subsection{Typing Environments}
\label{sec:typed-arith:type-env}

When working with languages keeping track of what elements in the language have what types is important.
For simple languages, such as the one introduced in this section, there is no need: All expressions have the same type.
However, in languages with variables, simple relations are not sufficient.
\emph{Typing Environments} are used to keep track of local variables and their types, and are explained in more depth in the next section.

Traditionally, typing environments are denoted by the greek letter $\Gamma$.
For the purposes of this section, modelling complete typing environments is not required and our typing environment $\Gamma$ just needs to nominally exist.

Using this notion of a type environment we can improve our definition of $\mathsf{WellTyped}$ to include triples in the form of $(\Gamma,e,T)$.
The $\mathsf{WellTyped}$ set will contain expressions that have a type $T$ derived from a local context $\Gamma$.
\[
(\Gamma,e,T)\in\mathsf{WellTyped}
\]
\noindent
To save on typing, the short hand $\Gamma\vdash e:T$ is used.
As our typing environment is empty the empty set symbol is used instead of $\Gamma$.

\subsection{Typing Rules}
\label{sec:typed-arith:rules}

Types and typing environments act as building blocks to help us construct well-typed programs.
To construct the set of relations for \textsf{WellTyped}, we need to define \emph{Typing Rules} that specify how expressions are typed and how types interact when expressions are combined.

\subsubsection{What are Typing Rules?}
\label{sec:typed-arith:rules:what}

Typing rules are a series of judgments that work in a particular context, with the top line defining the inputs and the bottom line the result.
The language in consideration in this section will have the following small set of typing rules.
Integers are given the type $\mathbb{Z}$.

\begin{prooftree}
\AxiomC{}
\LeftLabel{Integers}
\UnaryInfC{$\emptyset\vdash i : \mathbb{Z} $}
\end{prooftree}

\noindent
Negation is only applied to integers.
\begin{prooftree}
\AxiomC{$\emptyset\vdash e : \mathbb{Z}$}
\LeftLabel{Addition}
\UnaryInfC{$\emptyset\vdash -e : \mathbb{Z}$}
\end{prooftree}

\noindent
Operations only work with integers:

\begin{prooftree}
\AxiomC{$\emptyset\vdash e_1 : \mathbb{Z}$}
\AxiomC{$\emptyset\vdash e_2 : \mathbb{Z}$}
\LeftLabel{Addition}
\BinaryInfC{$\emptyset\vdash e_1+e_2 : \mathbb{Z}$}
\end{prooftree}

\begin{prooftree}
\AxiomC{$\emptyset\vdash e_1 : \mathbb{Z}$}
\AxiomC{$\emptyset\vdash e_2 : \mathbb{Z}$}
\LeftLabel{Subtraction}
\BinaryInfC{$\emptyset\vdash e_1-e_2 : \mathbb{Z}$}
\end{prooftree}

\begin{prooftree}
\AxiomC{$\emptyset\vdash e_1 : \mathbb{Z}$}
\AxiomC{$\emptyset\vdash e_2 : \mathbb{Z}$}
\LeftLabel{Multiplication}
\BinaryInfC{$\emptyset\vdash e_1*e_2 : \mathbb{Z}$}
\end{prooftree}

\begin{prooftree}
\AxiomC{$\emptyset\vdash e_1 : \mathbb{Z}$}
\AxiomC{$\emptyset\vdash e_2 : \mathbb{Z}$}
\LeftLabel{Division}
\BinaryInfC{$\emptyset\vdash e_1/e_2 : \mathbb{Z}$}
\end{prooftree}

\subsubsection{Modelling Typing Rules}
\label{sec:typed-arith:rules:modelling}

In non dependently typed languages we can \emph{model} typing rules through pattern matching.
For example here is na\"{i}ve implementation of \texttt{Addition}:
\begin{code}
addition : Arith -> Arith -> Maybe Arith
addition (Value a) (Value b) = Just (Value (a + b))
addition _         _         = Nothing
\end{code}

\noindent
However, there are two problems with this implementation.
First, this might be a bad implementation through the use of \texttt{Maybe}\footnote{
My programming foo is not strong with this.}
Secondly, it is rather verbose, and requires the creation of special functions to construct expressions; we have data constructors for that.

Here is a somewhat better implementation of the typing rules.
We leverage our ability to use dependent types, and embedd the typing rules directly within the constructors of the data type \texttt{Arith}

\begin{code}
data Arith : ArithTy -> Type where
  Value    : Int                            -> Arith TyValue
  Var      : String        -> Arith TyValue -> Arith TyValue
  Neg      : Arith TyValue                  -> Arith TyValue
  Addition : Arith TyValue -> Arith TyValue -> Arith TyValue
\end{code}

\noindent
Just look at that compact representation and emebedding of the typing rules.
Dependent types cool!


\paragraph{Note}
An alternate means to model this simple typed Arithmetic language is to introduce types for the operations.
This will allow for a more compact and stronger definition.
We leave this as an exercise for the reader.

\subsection{Denotational Semantics}
\label{sec:typed-arith:semantics}

So far we have introduced: how to model syntax, represent types, and enforce typing rules.
In this section we introduce the notion of \emph{Denotational Semantics} to provide an \emph{interpretation} of the language into \idris{}.
Denotational semantics is a technique that allows us to define the semantics of a language's expressions using another base notation, usually set-theory.
However, \idris{} is essentially applied maths\ldots and I am too lazy to provide a full denotational semantics for the language\footnote{This is left as an exercise for the author\ldots}.
Instead  we will provide an interpretation of in \idris{} such that we can build an interpreter that will allow us to interpret and execute programs.
Here the notation $\interpB{e}$ will be used to denote the interpretation of an element.

Alternatively, the definition of a language's semantics can be defined computationally using operational semantics\footnote{\textsc{Ibid}}.

\subsubsection{Interpreting Types}
\label{sec:typed-arith:semantics:types}

We begin by providing an interpretation of the types $T_{Arith}$:

\begin{center}
\begin{tabularx}{0.8\textwidth}{>{$}r<{$}>{\ttfamily}X}
\interpB{\mathbb{Z}}=& Int \\
\end{tabularx}
\end{center}

\noindent
A type interpreter can be written as follows:

\begin{code}
interpTy : ArithTy -> Type
interpTy TyValue = Int
\end{code}

\subsubsection{Expressions}
\label{sec:typed-arith:semantics:exrs}

Now we look to interpret expressions.

\begin{center}
\begin{tabularx}{0.8\textwidth}{>{$}r<{$}>{\ttfamily}X}
\interpB{i}     =& i\\
\interpB{-e}    =& (-1) * $\interpB{e}$\\
\interpB{x + y} =& $\interpB{x}$ + $\interpB{x}$ \\
\interpB{x - y} =& $\interpB{x}$ - $\interpB{x}$ \\
\interpB{x / y} =& $\interpB{x}$ `div` $\interpB{x}$ \\
\interpB{x * y} =& $\interpB{x}$ * $\interpB{x}$ \\
\end{tabularx}
\end{center}

\noindent
Raw values are directly translated into \idris{} values with type \texttt{Int}.
Negative numbers are interpreted expressions multiplied by $-1$.
Finally, the binary operations are mapped directly to their \idris{} equivalents.
Our interpreter for the language is thus constructed as follows:

\begin{code}
interp : Arith t -> interpTy t
interp (Val x)   = x
interp (Neg x)   = (-1) * (interp x)
interp (Add x y) = (interp x) + (interp y)
interp (Sub x y) = (interp x) - (interp y)
interp (Div x y) = (interp x) `div` (interp y)
interp (Mul x y) = (interp x) * (interp y)
\end{code}

\noindent
Not the similarities between the formalised semantics and representation in \idris{}.

\subsection{Running the Interpreter.}
\label{sec:running-interpreter}

Now that we have constructed the interpreter for the language we now can use the \texttt{interp} function to execute expressions.

\begin{code}
main : IO ()
main = do
  let expr = (Add (Val 1) (Val 2))
  print $ interp expr
\end{code}

%%% Local Variables:
%%% mode: latex
%%% TeX-master: "../tutorial.print"
%%% End:

%\section{Modelling Syntax}
\label{sec:notation}
\begin{center}\large\em
What do we say?
\end{center}

Like natural languages spoken by people today, programming languages have their own syntax and grammar rules.
These rules define the permissible expressions that are to be found within a language.
Fortunatly, programming languages are not as complex (i.e. are context free) as natural language and as such their syntax can be describe rather conscisly.
The most common notation used to present syntax is that of \emph{Backus-Naus Form} (BNF).
There are several varients of BNF that prove popular, for example: eBNF and aBNF.

A example of a BNF grammar to describe the syntax for a simple arithmetic language \allang{} is as follows:

\begin{bnf}
\bnfprod{AL}{%
  \bnfts{integer}
}\\
\bnfprod{AL}{%
  \bnfts{variable}
}\\
\bnfprod{AL}{%
  \bnfts{``-''}\bnfpn{AL}
}\\
\bnfprod{AL}{%
  \bnfpn{AL}\bnfts{``+''}\bnfpn{AL}
}\\
\bnfprod{AL}{%
  \bnfts{``(''}\bnfpn{AL}\bnfts{``)''}
}
\end{bnf}

\noindent
This language allows for expressions that describe integer operations addition subtraction, and modification of precedence using parentheses.
Notice how expressions are defined in terms of themselves.
Such grammars can be modelled within \idris{} as a simple data type.

\begin{code}
data Arith : Type where
  Value    : Int             -> Arith
  Var      : String -> Arith -> Arith
  Neg      : Arith           -> Arith
  Addition : Arith  -> Arith -> Arith
\end{code}

\noindent
Here we use data type constructors to model different expressions.
We also embedd simple interpretations of integers and variables.
Integers will be mapped to the \idris{} data type, and variables are represented with a name, and their value.
and symbol tables in which we can produce better representations of variables.
The syntax presented in the BNF grammar can be made more precise through use of an \emph{or} combinator.

\begin{bnf}
\bnfprod{AL}{%
  \bnfts{integer}
  \bnfor
  \bnfts{``-''}\bnfpn{AL}
  \bnfor
  \bnfpn{AL}\bnfts{``+''}\bnfpn{AL}
}
\end{bnf}

\noindent
Similar optimisations can be made to the \texttt{Arith} data type.

\begin{code}
data Arith = Value Int
           | Var String
           | Neg Arith
           | Addition Arith Arith
\end{code}

\noindent
\emph{Programming Language Theorists} (PLTs) are in the business of creating languages and BNF (and its popular variants) while descriptive and good to describe implementations can be too verbose.
PLTs use a particular variant of BNF in which the name of the language being defined (\allang{}) is replaced by a variable used to range over all the possible values of \allang{}.
The above BNF grammer would be re-expressed as follows:

\begin{bnf}
\bnfprod{$e$}{%
  \bnfts{$i$}
  \bnfor
  \bnfts{$x$}
  \bnfor
  \bnfts{$-e$}
  \bnfor
  \bnfts{$e+e$}
}
\end{bnf}

%%% Local Variables:
%%% mode: latex
%%% TeX-master: "tutorial.print"
%%% End:

%\section{Type System}
\label{sec:type}

\begin{quote}
A means to know what we say is correct when saying things.
\end{quote}

Grammars helps us define \emph{what we say}, and semantics helps us define \emph{what we mean when we say}.
Type systems helps us define \emph{a means to know what we say is correct}.
Types are used to helps us take note of the \emph{kind} of objects that will exist when the program is executed.
These objects will represent the types in our type system.
The set of types in a type system is often represented as $T$.
For the \allang{} language from earlier there is only one type of object in play integers $\mathbb{Z}$.

\begin{bnf}
\bnfprod{$T_{Arith}$}{%
  \bnfts{$\mathbb{Z}$}
}
\end{bnf}

\noindent
When modelling types in \idris{} a data \texttt{Ty} type can be constructed.
The type object for the \allang{} is simply:

\begin{code}
data ArithTy = TyValue
\end{code}

If there were more object types, for example anonymous functions we would extend $T_{Arith}$ to include said types.
For example, if the \allang{} language had both boolean and function types: $T_{Arith}$ would be as follows:

\begin{bnf}
\bnfprod{$T_{ALT}$}{%
  \bnfts{$\mathbb{Z}$}
  \bnfor
  \bnfts{$\mathbb{B}$}
  \bnfor
  \bnfts{$T_{ALT}\rightarrow T_{ALT}$}
}
\end{bnf}

\noindent
Where $\mathbb{B}$ represents objects of type Boolean, and $T\rightarrow T$ object that are simple one argument functions.
The function type indicates a mapping from the type of the arugment to the type of the return value.

The object \texttt{ArithTy} will be extend as follows:

\begin{code}
data ArithTy = TyValue | TyBool | TyFunc ArithTy ArithTy
\end{code}

\noindent
Note how we can model in \idris{} types and grammars in similar ways.

The purpose of a type system is to allow for the type of an expression to be calculated from the expression itself.
Continuting with the \allang{}, for example, the expression $e=(1 + 2)$ will have type $\mathbb{Z}$ as the result of evaluating $e$ will be $3$ which is an integer.

Type systems are defined using relations that will allow for the pairing of expressions to types.
We call this relation \textsf{WellTyped}, and will contain only: \emph{correctly typed expresions paired with their type}.
Thus:
\[
((1+2), Int)\in\mathsf{WellTyped}\\
(\text{``Bob''},Int)\notin\mathsf{WellTyped}
\]

Taking our \idris{} translations \textsf{WellTyped} can be represented as a list of expression type pairs.

\begin{code}
welltyped : List (Arith, ArithTy)
welltyped = [(Addition (Value 1) (Value 2), TyValue)]
\end{code}

Note we do not have a means (yet) to ensure that only well-typed expressions are constructed.
These are typing rules and are introduced in Section~\ref{sec:rules}.
Before we can specify typing rules, we need to introduce the notion of \emph{Typing Environments.}

\subsection{Environments}
\label{sec:type:env}

For simple expressions a simple set of relations is sufficient.
However, in languages with variables such a simple relations is not sufficient.
Variables are mutable and the type of a variables will not be known prior to the use of the variable, and may change during program execution.
To address this issue we introduce the concept of a \emph{typing environment}, which is defined as a set that contains tuples of variables and their types: $(x,T)$
Type environments can be extended, and the addition of new variables to the environment may result in old definitions being removed.
Traditionally, type environments are denoted by the greek letter $\Gamma$.
A simple definition for a typing environment for the \allang{} language in \idris{} is a list of variable name type pairs.
For readability we first construct a generic type alias \texttt{Context expr ty}, that specifies the variable  (\texttt{var}) and its current type (\texttt{ty}) of the language being represented.

\begin{code}
Context : Type -> Type -> Type
Context var ty = List (var, ty)
\end{code}

\noindent
Using this alias we can then create typing environments of type \texttt{Context}.
For example type environments in the \allang{} can be modelled as

\begin{code}
example_context : Context String ArithTy
example_context = [("foo", TyValue), ("bar", TyValue)]
\end{code}

\paragraph{Note} Here we are purposefuly using a \emph{named representation} and keeping track of variables using their actual name.
This makes it easier to understand what is going on within contexts.
An alternate would be to use a \emph{nameless representation}, in our larger case study (Section~\ref{}) we will show how a nameless representation using \emph{de Bruijn} indices works.


\begin{center}\bfseries
add notion of context per expression
\end{center}

\subsection{Operating on Environments}
\label{sec:type:env-operations}

Before type environments can be used in anger we define several operations that operate on $\Gamma$.

\subsubsection{Removing}
\label{sec:type:env-operations:remove}

The first operation will update $\Gamma$ by removing all references to $x$.
\[
\Gamma\backslash x
\]
\noindent
This operation can be implemented as follows:
\begin{code}
remove : (a,b) -> Context a b -> Context a b
remove e es = deleteBy (\(x,y),(c,d) => x==c) e es
\end{code}

\subsubsection{Extending}
\label{sec:type:env-operations:extend}

The second operations is used to extend $\Gamma$ with a variable $x$ and will possible overide previous definitions.
\[
\Gamma,x:T = (\Gamma\backslash x)\cup\{x,T\}
\]
\noindent
Which can be representened as:
\begin{code}
extend : (a,b) -> Context a b -> Context a b
extend e es = e :: (remove e es)
\end{code}

\subsubsection{Searching}
\label{sec:type:env-operations:lookup}

For completness, the final operation defined is used to search the typying environment to extract the type for a specified variable.
\[
\mathsf{lookup}(\Gamma,x:T) = \text{??}
\]
\noindent
This can be implemented as:
\begin{code}
lookup : a -> Context a b -> Maybe b
lookup = List.lookup
\end{code}

\subsubsection{Improving \idris{} Implementation}
\label{sec:type:env-operations:classes}

We can enforce our formal models in \idris{} better by creating a type class that groups our implementations together as follows:
\begin{code}
class (Eq a, Eq b) => TypeEnv a b where
  remove : (a,b) -> Context a b -> Context a b
  extend : (a,b) -> Context a b -> Context a b
  lookup : a     -> Context a b -> Maybe b

  remove e es = deleteBy (\(x,y),(c,d) => x==c) e es
  extend e es = e :: (remove e es)
  lookup = List.lookup
\end{code}

\subsubsection{Example}
\label{sec:types:example}

Explaining changes to a typing environment is difficult to see with type systems that only have a single type.
We further motivate the need for typing environments by using the alternate type system $T_{ALT}$ introduced earlier.
Thus given the following type environment:
\[
\Gamma_{1}=\{(foo,\mathbb{Z}),(g,\mathbb{Z}\rightarrow\mathbb{Z})\}
\]

\begin{code}
env : Context String ArithTy'
env = [("foo", TyValue'), ("g", TyFunc' TyValue' TyValue')]
\end{code}

\noindent
$\Gamma_{1}$ can be updated as follows:
\[
\Gamma_{1},g:\mathbb{Z} =\{(foo,\mathbb{Z}),(g,\mathbb{Z})\}
\]
\begin{code}
env' : TypeEnv String ArithTy' => Context String ArithTy'
env' = extend ("g", TyValue') env
\end{code}
\noindent
and:
\[
\Gamma_{1},f:(\mathbb{Z}\rightarrow\mathbb{Z})\rightarrow\mathbb{Z} =\{(foo,\mathbb{Z}),(g,\mathbb{Z}),(f,(\mathbb{Z}\rightarrow\mathbb{Z})\rightarrow\mathbb{Z})\}
\]
\begin{code}
env'' : TypeEnv String ArithTy' => Context String ArithTy'
env'' = extend ("f", TyFunc' (TyFunc' TyValue' TyValue')
                             TyValue')
               env'
\end{code}

\subsection{Improving the definition of \textsf{WellTyped}.}
\label{sec:types:example}

Using this notion of a type environment we can improve our definition of $\mathsf{WellTyped}$ to include triples in the form of $(\Gamma,e,T)$.
The $\mathsf{WellTyped}$ set will contain expressions that have a type $T$ derived from a local context $\Gamma$.
\[
(\Gamma,e,T)\in\mathsf{WellTyped}
\]
\noindent
To save on typing the short hand $\Gamma\vdash e:T$ is used.

\begin{code}
WellTyped : Type
WellTyped = List (Context Arith ArithTy , Arith, ArithTy)
\end{code}

%%% Local Variables:
%%% mode: latex
%%% TeX-master: "../tutorial.print"
%%% End:

%\section{Typing Rules}
\label{sec:rules:what}

\begin{quote}\em
  Specifying how expressions are to be typed.
\end{quote}

Types and typing environments act as building blocks to help us construct well-typed programs.
To construct the set of relations that for \textsf{WellTyped}, we need to define \emph{Typing Rules} that specify how expressions are typed and how types interact when expression are combined.

\subsection{What are Typing Rules?}
\label{sec:rules:what}

Typing rules are a series of judgments that work in a particular context, with the top line defining the inputs and the bottom line the result.
The \allang{} has the following small set of typing rules.
Integers are given the type $\mathbb{Z}$.

\begin{prooftree}
\AxiomC{}
\LeftLabel{Integers}
\UnaryInfC{$\Gamma\vdash i : \mathbb{Z} $}
\end{prooftree}

\noindent
Variables will have type $\mathbb{Z}$, only if the variable exists within $\Gamma$.
\begin{prooftree}
\AxiomC{$(x,T)\in\Gamma$}
\AxiomC{$T:\mathbb{Z}$}
\LeftLabel{Variables}
\BinaryInfC{$\Gamma\vdash i : \mathbb{Z}$}
\end{prooftree}

\noindent
Negation is only applied to integers.
\begin{prooftree}
\AxiomC{$\Gamma\vdash e : \mathbb{Z}$}
\LeftLabel{Addition}
\UnaryInfC{$\Gamma\vdash -e : \mathbb{Z}$}
\end{prooftree}

\noindent
Addition only happens on integers.
\begin{prooftree}
\AxiomC{$\Gamma\vdash e_1 : \mathbb{Z}$}
\AxiomC{$\Gamma\vdash e_2 : \mathbb{Z}$}
\LeftLabel{Addition}
\BinaryInfC{$\Gamma\vdash e_1+e_2 : \mathbb{Z}$}
\end{prooftree}

\noindent
The \allang{} is simple enough that the point of typing rules might be hard to see.
Imagine if \allang{} has the alternate typing system $T_{ALT}$ that includes boolean types.
The above typing rules still hold.
Addition will only work with integer values; expressions of type $\mathbb{B}$ cannot be used for addition.

\subsection{Modelling Typing Rules.}
\label{sec:rules:modelling}

In non dependently typed languages we can \emph{model} typing rules through pattern matching.
For example here is a typing rule for \texttt{Addition}
\begin{code}
addition : Arith -> Arith -> Maybe Arith
addition (Value a) (Value b) = Just (Value (a + b))
addition _         _         = Nothing
\end{code}

\noindent
However, there are two problems with this implementation.
First, this might be a bad implementation through the use of \texttt{Maybe}.
My programming foo is not strong with this.
Secondly, the typing context $\Gamma$ has not been taken into account.
Without the typing context we don't know much.

Before we introduce a typing context into our implemenation.
Here is a better somewhat na\"{i}ve implementation of the typing rules.
We leverage our ability to use dependenty types, and embedd the type of each expression directly within the constructors of a redefined data type of $AL$.

\begin{code}
data Arith : ArithTy -> Type where
  Value    : Int                            -> Arith TyValue
  Var      : String                         -> Arith TyValue
  Neg      : Arith TyValue                  -> Arith TyValue
  Addition : Arith TyValue -> Arith TyValue -> Arith TyValue
\end{code}

\noindent
Just look at that compact representation and emebedding of the typing rules.
Dependent types cool!

Now we look to embedd a typing context $\Gamma$.
Recall from Section~\ref{}, that every expression needs to be assoicated with a type environment.
Thus as well as the type of each expression we embedd an instantiation of the type environment \texttt{env} with every expression.

\begin{center}
  \bfseries How to enforce variables? Is \texttt{HasType} the correct way?
\end{center}

%%% Local Variables:
%%% mode: latex
%%% TeX-master: "../tutorial.print"
%%% End:

%\section{Denotational Semantics}
\label{sec:denote}

\begin{center}\large\em
Using another base to say what we mean when we say something.
\end{center}

\noindent
So far we have introduced: how to model syntax, represent types, and enforce typing rules.
In this section we introduce the notion of \emph{Denotational Semantics} to provide an \emph{interpretation} of \allang{} to \idris{}.
Denotational semantics is a technique that allows us to define the semantics of a language's expressions using mathematics, usually set-theory.
However, maths while important can be boring, and I wouldn't know how to provide a denotational semantics for \allang{}\footnote{This is left as an exercise for the author\ldots}.
Instead  we will provide an interpretation of \allang{} in \idris{} such that we can build an interpreter that will allow us to interpret and execute \allang{} programs.
Here the notation $\interpB{e}$ will be used to denote the interpretation of an element in \allang{} to \idris{}.

An alternate definition of a language's semantics can be defined computationally using operational semantics\footnote{\textsc{Ibid}}.


\subsection{Types}
\label{sec:denote:types}

We begin by providing an interpretation of the types $T_{Arith}$:

\begin{center}
\begin{tabularx}{0.8\textwidth}{>{$}r<{$}>{\ttfamily}X}
\interpB{\mathbb{Z}}=& Int \\
\end{tabularx}
\end{center}

\noindent
A type interpreter can be written as follows:

\begin{code}
interpTy : ArithTy -> Type
interpTy TyValue = Int
\end{code}

\noindent
Simple, a little too simple methings\ldots.
Thus, we further motivate the interpretation of types using the alternate set of types $T_{ALT}$:

\begin{center}
\begin{tabularx}{0.8\textwidth}{>{$}r<{$}>{\ttfamily}X}
\interpB{\mathbb{Z}}=& Int \\
\interpB{\mathbb{B}}=& Bool \\
\interpB{T_{A}\rightarrow T_{B}}=& $\interpB{T_{A}}$ -> $\interpB{T_{B}}$ \\
\end{tabularx}
\end{center}

\noindent
Remember, types in \idris{} are first class values, and can be presented like any other values.
Yay for dependent types!
A type interpreter can be written as follows:

\begin{code}
interpTy : ArithTy -> Type
interpTy TyValue              = Int
interpTy TyBool               = Bool
interpTy (TyFunc argty retty) = interpTy argty -> interpTy retty
\end{code}

\subsection{Expressions}
\label{sec:denote:expressions}

Now we look to dealing with expressions.

\begin{center}
\begin{tabularx}{0.8\textwidth}{>{$}r<{$}>{\ttfamily}X}
\interpB{i}     =& i\\
\interpB{x}     =& ??\\
\interpB{-e}    =& (*) (-1) $\interpB{e}$\\
\interpB{x + y} =& (+) $\interpB{x}$ $\interpB{x}$ \\
\end{tabularx}
\end{center}

\noindent
Raw values are directly translated into \idris{} values with type \texttt{Int}.
Variables are\ldots
Negative numbers are interpreted expressions multiplied by $-1$.
Finally,  addition of two expressions is mapped to the \idris{} addition primitive \texttt{(+)}.
Our interpreter for \allang{} is contructed as follows:

\begin{code}
interp : Env ArithG -> Arith ArithG ArithTy -> interpTy ArithTy
interp env (Val x)        = x
interp env (Var name x)   = ?todo
interp env (Neg x)        = (*) (-1)           (interp env x)
interp env (Addition x y) = (+) (interp env x) (interp env y)
\end{code}

\noindent
Where:
\begin{code}
ArithG : Type
ArithG = Context Arith ArithTy
\end{code}


\subsection{Running}
\label{sec:denote:running}

\begin{center}
  \large\em Examples
\end{center}

\begin{center}
  \large\em How to run the interpreter.
\end{center}

%%% Local Variables:
%%% mode: latex
%%% TeX-master: "../tutorial.print"
%%% End:


\newpage
\printbibliography{}
\newpage
\glsaddall
\printglossary[style=listdotted]
\end{document}
